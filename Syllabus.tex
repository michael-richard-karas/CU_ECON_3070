\documentclass[11pt]{article}
%\voffset -.8in \hoffset -.7in \textheight 9.9in \textwidth 8.8in
\usepackage{geometry,url}
\geometry{body={7.0in,9.5in}, centering}
\usepackage{multicol}
\usepackage{hyperref}
\usepackage{tabularx}
\usepackage{ulem}
\usepackage{fancyhdr}
\pagestyle{fancy}
\pagenumbering{arabic}
\begin{document}
\begin{center}
\textbf{\Large Intermediate Microeconomic Theory}\\
The University of Colorado, Boulder\\ Economics Department\\ ECON-3070-030
\end{center}
%\framebox[6.2in]{\makebox[\totalheight]}

%\begin{tabular}{l@{\qquad}l@{\qquad}l@{\qquad}}
 %   Course Title: & &  \\
  %  \end{tabular}
---------------------------------------------------------------------------------------------------------------------------------
\vspace{-.1in}
\begin{center}
\begin{tabular}{llll}
    \textbf{Instructor:} & \footnotesize{Michael Karas}  &   \textbf{Lecture Location:}  & \footnotesize{ECON 119}\\ 
    \textbf{Email:} & \footnotesize{michael.karas@colorado.edu} &   \textbf{Lecture Day/Time:} & \footnotesize{MWF 2:30pm - 3:20pm}\\
    \textbf{Office Hours:} & \footnotesize{}   &  \textbf{Recitation 031 Location:} & \footnotesize{CLRE W262} \\
    \textbf{} & \footnotesize{}   &  \textbf{Recitation 031 Day/Time:} & \footnotesize{Mon 9:05am - 9:55am} \\
    \textbf{Office Location:} & \footnotesize{}  & \textbf{Recitation 032 Location:} & \footnotesize{ECCR 1B51}\\
    \textbf{Teaching Assistant:} & \footnotesize{}   & \textbf{Recitation 032 Day/Time:} & \footnotesize{Wed 10:10am - 11:00am}\\
    \textbf{TA's Email:} & \footnotesize{}   & \textbf{Term:} & \footnotesize{Spring 2025}\\
    \textbf{TA's Office Hours:} & \footnotesize{}   & \textbf{Credit Hours:} & \footnotesize{4}\\
    \textbf{} & \footnotesize{}   &  \textbf{} & \footnotesize{} \\
        \textbf{TA's Office Location:} & \footnotesize{}   & \textbf{} & \\
\end{tabular}
\end{center}

---------------------------------------------------------------------------------------------------------------------------------

\noindent{\bf COURSE DESCRIPTION}\\
Explores theory and application of models of consumer choice, firm and market organization, and general equilibrium.  Extensions include intertemporal decisions, decisions under uncertainty, externalities, and strategic interaction.\\

\noindent {\bf COURSE OBJECTIVES}\\
Upon successful completion of this class a student should be able to:
\begin{itemize}
\item {\bf\emph{Apply Intermediate Level Concepts}}:  Demonstrate a solid understanding of intermediate microeconomic concepts, including consumer choice theory, producer behavior, market equilibrium, and welfare analysis.\vspace{-.1in}
\item {\bf\emph{Evaluate Consumer Choice}}:  Evaluate consumer preferences, utility maximization, and demand theory, and utilize indifference curve analysis to understand how changes in prices and income impact consumer choices.\vspace{-.1in}
\item {\bf\emph{Assess Producer Decision-making}}:  Analyze producer decisions regarding profit maximization using concepts such as production functions, cost curves, and revenue curves.\vspace{-.1in}
\item {\bf\emph{Analyze Market Structures}}:  Analyze and differentiate between various market structures, including perfect competition, monopoly, monopolistic competition, and oligopoly, considering their implications for pricing, output, and economic efficiency.\vspace{-.1in} 
\item {\bf\emph{Employ Mathematical Techniques}}:  Utilize mathematical techniques, such as calculus and algebra, to analyze and solve complex microeconomic problems, including optimization and equilibrium calculations.
\end{itemize}

\noindent{\bf PREREQUISITES}\\
A minimum grade of C- in ECON 2010.  A minimum grade of C- in one of the following courses: ECON 1088, MATH 1081, MATH 1300, MATH 1310, MATH 1330, the two course sequence: APPM 1340/APPM 1345, APPM 1350, or FNCE 2010.\\  

\noindent{\bf TEXTBOOK(\sl optional):}\\
{\sl \ Microeconomics, 6th Edition}, by Devid Besanko and Ronald Braeutigam (ISBN:978-1-119-55484-4), Wiley. 2020.\\
Students may optionally purchase the textbook.  Owning the textbook is not required for any graded assignments in the course.  The class will follow the textbook closely so students may find reviewing the chapters and practice problems useful when studying for exams.\\
\newpage
\noindent{\bf GRADING}\\
\begin{center}
\begin{tabular}{lc}
    \hline
    Grading Scheme \\
    \hline
    Attendance & 5\% \\
    Recitations & 10\% \\
    Problem Sets & 15\% \\
    Midterm I & 17.5\% \\
    Midterm II & 17.5\% \\
    Midterm III & 17.5\% \\
    Final Exam & 17.5\% \\
\end{tabular}
\end{center}
Letter grades will be assigned with a "+/-" according to the following chart.  Following guidelines set out by the Economics Department, the overall average grade of the course will be curved to a B-/C+ (around 80\%).  The size of this curve will not be known until after all components of the course have been graded. \\
\begin{center}
\begin{tabular}{|l|c|}
    \hline
    Grade & Percentage \\
    \hline
    A & 94 \& Above \\
    A- & 90-93.99\% \\
    B+ & 87-89.99\% \\
    B & 83-86.99\% \\
    B- & 80-82.99\% \\
    C+ & 77-79.99\% \\
    C & 73-76.99\% \\
    C- & 70-72.99\% \\
    D+ & 67-69.99\% \\
    D & 63-66.99\% \\
    D- & 60-62.99\% \\
    F & Below 60\% \\
    \hline
\end{tabular}
\end{center}

\noindent{\bf EXAMS}\\\begin{center}
\begin{tabular}{ccc}
    Midterm I & February 19th (Wed) & Chapters 1-5 \\
    Midterm II & March 19th (Wed) & Chapters 6-8 \\
   	Midterm III & April 23rd (Wed) & Chapters 9-13 \\
    Final Exam & May 4th (Sun) & Cumulative \\
\end{tabular}
\end{center}
The two midterms will be held during regular lecture time in the lecture classroom (GUGG 205) on October 4th and November 8th.  The midterms will not be cumulative, but will consist of the content covered in the indicated chapters.  The exams are structured with both multiple choice and short answer questions.  \uline{There will be no make-up exams}.  If you miss a midterm due to a proven emergency or other unusual circumstances (you should inform me prior to each midterm), the weight of the midterm will be shifted to the final exam (\emph{e.g.} if you miss a midterm I will re-weight the final exam to be 45\% of your final grade).  The final exam will be held in the lecture classroom (GUGG 205) on Sunday, December 17th.  It will cumulative and consist of multiple choice and short answer questions.  Moverover, \uline{you must take the final exam in this course: there will no make-up final exam.} \\

\newpage
\noindent{\bf ATTENDANCE}\\
5\% of your final grade will be determined by attendance.  At the beginning of each lecture I will have an attendance sign-in sheet at the front of class.  As you come into class please sign your name so you get credit for attendance.  If you arrive late to class, you must wait until the end of lecture to sign-in.  In order to allow for some leeway, you only need to be counted for 85\% of the total lectures to receive the full 5\% for your attendance grade.  If you attend fewer than 85\% of the total lectures, your attendance grade will be the proportion of the lectures you did attend.\\

\noindent{\bf PROBLEM SETS}\\
Problem sets will serve as a primary component of the course for you to test your knowledge of the material before exams.  You will receive feedback on your answers from the problem sets.  They will be due one week after they are assigned.  You are allowed to work with other students in the course on the problem sets, but each student must turn in their own set of answers.  \uline{A physical copy of your answers must be turned in during the lecture that they are due.  Late problem sets will not be accepted}.  There will be 6 total problem sets through the semester.  In order to allow for some leeway, your lowest graded problem set will be dropped.\\ 

\noindent{\bf RECITATION}\\
Starting the second week of the semester your will have a weekly recitation session with the TA for this course.  The goal of recitation is for you to review, practice, and have an opportunity to ask questions about the material covered in the course.  Your TA will determine exactly how recitations are run and how they will be graded, however attendance will be a primary component of the final recitation grade.\\

\noindent{\bf COMMUNICATION}\\
Outside of lectures and office hours, the optimal way to contact me would be through email.  Please use your colorado.edu account to email me because sometimes other email accounts are filtered into my spam folder.  Please allow 24 hours for a response during the week, and 48 hours on weekends.  \uline{I am not allowed, by law, to discuss grades with you or anyone else via email.  Please do not email me with questions about your grades: instead, see me during office hours}.\\

\noindent{\bf MISSED ASSIGNMENTS}\\
In general, late assignments will not be accepted.  \uline{In the case that you miss an assignment due to an extreme illness, I will shift the weight of the missed one to subsequent assignments.  If you miss a midterm exam the weight will be shifted to the final exam.  Proper documentation will be required for all missed assignments and exams.}\\

\noindent{\bf EXAM POLICIES}\\
Exams are the only components of the final grade for which collaboration is explicitly prohibited.  If you are caught cheating you will receive a 0 on the associated assignment.  Non-graphing calculators are allowed to be used during exams; use of any other device will be considered cheating.\\

\noindent{\bf ELECTRONICS}\\
Cell phone use is not permitted during class.  If there is an emergency requiring the use of a cell phone please step outside the classroom.  The use of tablets and laptops for educational purposes (i.e. lecture slides, Canvas, note taking) is permitted; however, I reserve the right to disallow their use if electronics are not being used appropriately.\\

\newpage
\noindent{\bf TENTATIVE COURSE SCHEDULE}\\

\begin{tabularx}{\textwidth}{|c|c|c|X|}
    \hline
    \textbf{Week} & \textbf{Dates} & \textbf{Chapters Covered} & \textbf{Special Notes} \\
    \hline
    Week 1 & Jan 13, Jan 15, Jan 17 & Ch. 1, Math Review, \& Ch. 2 & \\
    \hline
    Week 2 & Jan 22, Jan 24 & Ch. 2 \& 3 & Jan 20 MLK Day\\
    \hline
    Week 3 & Jan 27, Jan 29, Jan 31 & Ch. 3 \& 4 & \\
    \hline
    Week 4 & Feb 3, Feb 5, Feb 7 & Ch. 4 \& 5 & PS1 due Jan 29 \\
    \hline
	Week 5 & Feb 10, Feb 12, Feb 14 & Ch. 5 \& 6 & PS2 due Feb 12\\
    \hline
    Week 6 & Feb 17, Feb 19, Feb 21 & Ch. 6 \& 7 & Midterm I Feb 19 \\
    \hline
    Week 7 & Feb 24, Feb 26, Feb 28 & Ch. 7 & \\
    \hline
    Week 8 & Mar 3, Mar 5, Mar 7 & Ch. 8 \& 9 & PS3 due Mar 5 \\
    \hline
	Week 9 & Mar 10, Mar 12, Mar 14 & Ch. 9 \& 10 & PS4 due Mar 12 \\
    \hline
	Week 10 & Mar 17, Mar 19, Mar 21 & Ch. 10 \& 11 & Midterm II Mar 19\\
    \hline
	Week 11 & & Ch. 11 \& 12 & Mar 24 - 28 Spring Break\\
    \hline
	Week 12 & Mar 31, Apr 2, Apr 4 & Ch. 13 & \\
    \hline
	Week 13 & Apr 7, Apr 9, Apr 11 & & PS5 due Apr 9 \\
    \hline
	Week 14 & Apr 14, Apr 16, Apr 18 & Ch. 14 \& Ch. 15 & PS6 due Apr 16 \\
    \hline
	Week 15 & Apr 21, Apr 23, Apr 25 & Ch. 15 \& Ch. 16 & Midterm III Apr 23\\
    \hline
	Week 16 & Apr 28, Apr 30 & Ch. 14 & May 2 Reading Day\\
    \hline
	Week 17 & & & Sun, May 4 Final Exam (1:30pm - 4:00pm)\\
    \hline
\end{tabularx}

\newpage

\medskip\noindent{\bf UNIVERSITY POLICIES:}\\

\noindent{\bf CLASSROOM BEHAVIOR}\\
Students and faculty are responsible for maintaining an appropriate learning environment in all instructional settings, whether in person, remote, or online. Failure to adhere to such behavioral standards may be subject to discipline. Professional courtesy and sensitivity are especially important with respect to individuals and topics dealing with race, color, national origin, sex, pregnancy, age, disability, creed, religion, sexual orientation, gender identity, gender expression, veteran status, political affiliation, or political philosophy.\\
\\
For more information, see the \href{https://www.colorado.edu/policies/student-classroom-course-related-behavior}{classroom behavior policy}, \href{https://www.colorado.edu/sccr/student-conduct}{the Student Code of Conduct} and \href{https://www.colorado.edu/oiec/}{the Office of Institutional Equity and Compliance}.\\

\noindent{\bf REQUIREMENTS FOR INFECTIOUS DISEASES}\\
Members of the CU Boulder community and visitors to campus must follow university, department, and building health and safety requirements and all public health orders to reduce the risk of spreading infectious diseases. \\
\\
The CU Boulder campus is currently mask optional. However, if masks are again required in classrooms, students who fail to adhere to masking requirements will be asked to leave class. Students who do not leave class when asked or who refuse to comply with these requirements will be referred to Student Conduct & Conflict Resolution. Students who require accommodation because a disability prevents them from fulfilling safety measures related to infectious disease will be asked to follow the steps in the "Accommodation for Disabilities" statement on this syllabus.\\
\\
For those who feel ill and think you might have COVID-19 or if you have tested positive for COVID-19, please stay home and follow the \href{https://www.colorado.edu/healthcenter/coronavirus-updates/symptoms-and-what-do-if-you-feel-sick}{further guidance of the Public Health Office}. For those who have been in close contact with someone who has COVID-19 but do not have any symptoms and have not tested positive for COVID-19, you do not need to stay home. \\

\noindent{\bf ACCOMODATIONS FOR DISABILITIES, TEMPORARY MEDICAL CONDITIONS, AND MEDICAL ISOLATION}\\
\href{https://www.colorado.edu/disabilityservices/}{Disability Services} determines accommodations based on documented disabilities in the academic environment. If you qualify for accommodations because of a disability, submit your accommodation letter from Disability Services to your faculty member in a timely manner so your needs can be addressed. Contact Disability Services at 303-492-8671 or dsinfo@colorado.edu for further assistance.\\
\\
If you have a temporary medical condition or required medical isolation for which you require accommodation, please let me know ahead of the scheduled lecture or recitation.  I will handle each situation on a case by case basis, but the most common accommodation will be excusing your attendance for the recitations that you must miss so they do not count against your attendance grade.  Also see \href{https://www.colorado.edu/disabilityservices/students/temporary-medical-conditions}{Temporary Medical Conditions} on the Disability Services website.\\

\noindent{\bf PREFERRED STUDENT NAMES AND PRONOUNS}\\
CU Boulder recognizes that students' legal information doesn't always align with how they identify. Students may update their preferred names and pronouns via the student portal; those preferred names and pronouns are listed on instructors' class rosters. In the absence of such updates, the name that appears on the class roster is the student's legal name.\\
\newpage
\noindent{\bf HONOR CODE}\\
All students enrolled in a University of Colorado Boulder course are responsible for knowing and adhering to the \href{https://www.colorado.edu/sccr/honor-code}{Honor Code}. Violations of the Honor Code may include but are not limited to: plagiarism (including use of paper writing services or technology [such as essay bots]), cheating, fabrication, lying, bribery, threat, unauthorized access to academic materials, clicker fraud, submitting the same or similar work in more than one course without permission from all course instructors involved, and aiding academic dishonesty.\\
\\
All incidents of academic misconduct will be reported to Student Conduct & Conflict Resolution: honor@colorado.edu, 303-492-5550. Students found responsible for violating the {https://www.colorado.edu/sccr/honor-code}{Honor Code} will be assigned resolution outcomes from the Student Conduct & Conflict Resolution as well as be subject to academic sanctions from the faculty member. Visit {https://www.colorado.edu/sccr/honor-code}{Honor Code} for more information on the academic integrity policy.\\

\noindent{\bf SEXUAL MISCONDUCT, DISCRIMINATION, HARRASMENT, AND/OR RELATED RETALIATION}\\
CU Boulder is committed to fostering an inclusive and welcoming learning, working, and living environment. University policy prohibits \href{https://www.colorado.edu/oiec/policies/discrimination-harassment-policy/protected-class-definitions}{protected-class} discrimination and harassment, sexual misconduct (harassment, exploitation, and assault), intimate partner violence (dating or domestic violence), stalking, and related retaliation by or against members of our community on- and off-campus. These behaviors harm individuals and our community. The Office of Institutional Equity and Compliance (OIEC) addresses these concerns, and individuals who believe they have been subjected to misconduct can contact OIEC at 303-492-2127 or email cureport@colorado.edu. Information about university policies, \href{https://www.colorado.edu/oiec/reporting-resolutions/making-report}{reporting options}, and support resources can be found on the \href{https://www.colorado.edu/oiec/}{OIEC website}.\\
\\
Please know that faculty and graduate instructors have a responsibility to inform OIEC when they are made aware of incidents related to these policies regardless of when or where something occurred. This is to ensure that individuals impacted receive an outreach from OIEC about their options for addressing a concern and the support resources available. To learn more about reporting and support resources for a variety of issues, visit \href{https://www.colorado.edu/dontignoreit/}{Don’t Ignore It}.\\

\noindent{\bf RELIGIOUS HOLIDAYS}\\
Campus policy regarding religious observances requires that faculty make every effort to deal reasonably and fairly with all students who, because of religious obligations, have conflicts with scheduled exams, assignments or required attendance. In this class, please let me know ahead of any conflict with class assignments and tests and we can discuss how to rearrange those to accommodate your religious observance.\\
\\
See the \href{https://www.colorado.edu/policies/observance-religious-holidays-absences-classes-or-exams}{campus policy regarding religious observances} for full details.\\

\noindent{\bf MENTAL HEALTH AND WELLNESS}\\
The University of Colorado Boulder is committed to the well-being of all students. If you are struggling with personal stressors, mental health or substance use concerns that are impacting academic or daily life, please contact \href{https://www.colorado.edu/counseling/}{Counseling and Psychiatric Services (CAPS)} located in C4C or call (303) 492-2277, 24/7.\\
\\
Free and unlimited telehealth is also available through \href{https://www.colorado.edu/health/academiclivecare}{Academic Live Care}. The Academic Live Care site also provides information about additional wellness services on campus that are available to students.


\end{document}
