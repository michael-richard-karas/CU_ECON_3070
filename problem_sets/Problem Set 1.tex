\documentclass[11pt]{article}
%% Language and font encodings
\usepackage[english]{babel}
\usepackage[utf8x]{inputenc}
\usepackage[T1]{fontenc}

\usepackage{helvet}

%% Sets page size and margins
\usepackage[letterpaper,top=3cm,bottom=2cm,left=3cm,right=3cm,marginparwidth=1.75cm]{geometry}

%% Useful packages
\usepackage{amsmath}
\usepackage{graphicx}
\usepackage{tcolorbox}
\usepackage{amssymb}
\usepackage{amsthm}
\usepackage{lastpage}
\usepackage{accents}
\usepackage{multicol}
\usepackage{booktabs}

% For better list numbering
\usepackage[shortlabels]{enumitem}

% Font
% \usepackage{tgbonum}


% Tikz
\usepackage{tikz}

\usetikzlibrary{calc,fit,shapes.misc,backgrounds}
\usepackage{pgfplots}
\pgfplotsset{compat = newest}
\usetikzlibrary{positioning, arrows.meta}
\usepgfplotslibrary{fillbetween}

% Headers
\usepackage{fancyhdr}
\pagestyle{fancy}

% Store \@title as \thetitle
\makeatletter
\let\thetitle\@title
\makeatother

\fancyhf{}
\lhead{\fontfamily{qbk}\fontsize{10}{11}\selectfont Name:}
\rhead{\fontfamily{qbk}\fontsize{10}{11}\selectfont ECON 3070-030}
\rfoot{\fontfamily{qbk}\fontsize{10}{11}\selectfont \thepage}


% Sections and Subsections

% define colors
\definecolor{buff-gold}{HTML}{CFB87C}
\definecolor{buff-grey}{HTML}{565A5C}
% custom tcolorbox
\tcbset{colframe=buff-gold, colback=white!100!black}

% new page per section
\usepackage{titlesec}
\newcommand{\sectionbreak}{\clearpage}
% change style of section
\usepackage{sectsty}
\sectionfont{\color{buff-gold} \fontfamily{qbk}\selectfont}
\subsectionfont{\color{buff-grey} \fontfamily{qbk}\selectfont}


\begin{document}
  
\section*{Problem Set 1}

\begin{enumerate}

	\item Suppose the market demand curve for a product is given by 
	$$
		Q^D_{mkt} = 1000 - 10P
	$$
	
	and the market supply curve is given by
	$$
		Q^S_{mkt} = -50 + 25P
	$$
	\begin{enumerate}
    \item What are the equilibrium price and quantity
    
    \item What is the inverse form of the demand curve?
  \end{enumerate}
  
  \newpage 
  \item Suppose that the individual demand curves for two individuals are given by
  $$
    Q^D_1 = \begin{cases} 
      200 - 10P & \text{if } P \leq 20 \\ 
      0         & \text{otherwise} 
    \end{cases}
    \quad\text{ and }\quad
    Q^D_2 = \begin{cases} 
      100 - 10P & \text{if } P \leq 10 \\ 
      0         & \text{otherwise} 
    \end{cases}
  $$
  and that the supply curve for the firm is given by
  $$
    Q^S_{mkt} = \begin{cases} 
      20P - 100 & \text{if } P \geq 5 \\ 
      0         & \text{otherwise} 
    \end{cases}.
  $$
  Find the market equilibrium (keeping in mind that the market price may be such that only 1 consumer will be willing to stay in the market).

  \newpage 
  \item Suppose that when Snarfburger originally charged a price of $\$5$ for their burger, they sold $1,000$ burgers per week. Thinking that they could potentially make more money by charging a higher price, they raised their price by $50$ cents. After raising their price, they sold $800$
  units per week. 
  
  \begin{enumerate}
    \item Find the price elasticity of demand for Snarfburgers. 
    
    \item Write a sentence interpreting the price elasticity of demand you calculated.
  \end{enumerate}
  
  \newpage
\begin{table}[htbp]
    \centering
    \caption{Estimates of the Price Elasticity of Demand for Selected Food Products}
    \resizebox{\textwidth}{!}{
    \begin{tabular}{lc}
        \toprule
        Product & Estimated $\epsilon_{Q,P}$ \\
        \midrule
        Cigars & -0.756 \\
        Canned and cured seafood & -0.736 \\
        Fresh and frozen fish & -0.695 \\
        Cheese & -0.595 \\
        Ice cream & -0.349 \\
        Beer and malt beverages & -0.283 \\
        Bread and bakery products & -0.220 \\
        Wine and brandy & -0.198 \\
        Cookies and crackers & -0.188 \\
        Roasted coffee & -0.120 \\
        Cigarettes & -0.107 \\
        Chewing tobacco & -0.105 \\
        Pet food & -0.061 \\
        Breakfast cereal & -0.031 \\
        \bottomrule
        \multicolumn{2}{p{\textwidth}}{\footnotesize Pagoulatos \& Sorensen (1986)}\\
    \end{tabular}
    }
\end{table}
	\begin{enumerate}
    \item Which good is the most inelastic?
    
    \item Which food product is more inelastic, Cheese or Roasted coffee?
    
    \item Is Ice cream considered an elastic good?
    
    \item Write a sentence interpreting the elasticity for Cookies and Crackers.
  \end{enumerate}
  
  
\newpage
  \item With peanut butter on the $x$-axis, and jelly on the $y$-axis, draw a set of at least two indifference curves to represent the following types of preferences:

  \begin{enumerate}
    \item I like both peanut butter and jelly, and always get the same additional satisfaction froman ounce of peanut butter as I do from 2 ounces of jelly.
    \item I like peanut butter, but neither like nor dislike jelly.
    \item I like peanut butter, but dislike jelly.
    \item I like peanut butter and jelly, but I only want 2 ounces of peanut butter for every ounce of jelly.
  \end{enumerate}

  \vspace*{20mm}
  \begin{multicols}{2}
    \begin{tikzpicture}
      \begin{axis}[
        width = 6cm,
        height = 6cm,
        xmin = 0, xmax = 10,
        ymin = 0, ymax = 10,
        axis lines = left,
        xtick = \empty, ytick = \empty,
        x label style={at={(axis description cs:0.5,-0.07)},anchor=north},
        y label style={at={(axis description cs:-0.07,.5)},anchor=south},
        xlabel = {\small Peanut Butter},
        ylabel = {\small Jelly},
        clip = false,
      ]
      \end{axis}
    \end{tikzpicture}

    \vspace{20mm}      
    \begin{tikzpicture}
      \begin{axis}[
        width = 6cm,
        height = 6cm,
        xmin = 0, xmax = 10,
        ymin = 0, ymax = 10,
        axis lines = left,
        xtick = \empty, ytick = \empty,
        x label style={at={(axis description cs:0.5,-0.07)},anchor=north},
        y label style={at={(axis description cs:-0.07,.5)},anchor=south},
        xlabel = {\small Peanut Butter},
        ylabel = {\small Jelly},
        clip = false,
      ]
      \end{axis}
    \end{tikzpicture}

    \vspace{20mm}
    \begin{tikzpicture}
      \begin{axis}[
        width = 6cm,
        height = 6cm,
        xmin = 0, xmax = 10,
        ymin = 0, ymax = 10,
        axis lines = left,
        xtick = \empty, ytick = \empty,
        x label style={at={(axis description cs:0.5,-0.07)},anchor=north},
        y label style={at={(axis description cs:-0.07,.5)},anchor=south},
        xlabel = {\small Peanut Butter},
        ylabel = {\small Jelly},
        clip = false,
      ]
      \end{axis}
    \end{tikzpicture}

    \vspace{20mm}
    \begin{tikzpicture}
      \begin{axis}[
        width = 6cm,
        height = 6cm,
        xmin = 0, xmax = 10,
        ymin = 0, ymax = 10,
        axis lines = left,
        xtick = \empty, ytick = \empty,
        x label style={at={(axis description cs:0.5,-0.07)},anchor=north},
        y label style={at={(axis description cs:-0.07,.5)},anchor=south},
        xlabel = {\small Peanut Butter},
        ylabel = {\small Jelly},
        clip = false,
      ]
      \end{axis}
    \end{tikzpicture}
  \end{multicols}

  \newpage
  \item For each of the utility functions below, answer the following questions.
  
  \begin{enumerate}[(i)]
    \item Is the assumption that more is better satisfied for both goods?
    \item What is the marginal utility for each of the goods?
    \item Does the marginal utility of x diminish, remain constant, or increase as the consumer buys more $x$? Explain.
    \item What is the marginal rate of substitution ($MRS$) of $x$ for $y$?
    \item Is the $MRS_{x,y}$ diminishing, constant, or increasing as the consumer substitutes more $x$ for $y$ along an indifference curve?
  \end{enumerate}


    \begin{enumerate}[(a)]
      \item $U(x,y) = 10x - 0.5x^2 + 20y - y^2$
      
      \vspace*{80mm}
      \item $U(x,y) = 6x^{1/3}y^{2/3}$
      
      \vspace*{80mm}
      \item $U(x,y) = xy^2$
      
      \vspace*{100mm}
      \item $U(x,y) = \sqrt{x} + 2\sqrt{y}$
    \end{enumerate}
\end{enumerate}

\end{document}