\documentclass[11pt]{article}
%% Language and font encodings
\usepackage[english]{babel}
\usepackage[utf8x]{inputenc}
\usepackage[T1]{fontenc}

\usepackage{helvet}

%% Sets page size and margins
\usepackage[letterpaper,top=3cm,bottom=2cm,left=3cm,right=3cm,marginparwidth=1.75cm]{geometry}

%% Useful packages
\usepackage{amsmath}
\usepackage{graphicx}
\usepackage{tcolorbox}
\usepackage{amssymb}
\usepackage{amsthm}
\usepackage{lastpage}
\usepackage{accents}
\usepackage{multicol}
\usepackage{booktabs}

% For better list numbering
\usepackage[shortlabels]{enumitem}

% Font
% \usepackage{tgbonum}


% Tikz
\usepackage{tikz}

\usetikzlibrary{calc,fit,shapes.misc,backgrounds}
\usepackage{pgfplots}
\pgfplotsset{compat = newest}
\usetikzlibrary{positioning, arrows.meta}
\usepgfplotslibrary{fillbetween}

% Headers
\usepackage{fancyhdr}
\pagestyle{fancy}

% Store \@title as \thetitle
\makeatletter
\let\thetitle\@title
\makeatother

\fancyhf{}
\lhead{\fontfamily{qbk}\fontsize{10}{11}\selectfont Name:}
\rhead{\fontfamily{qbk}\fontsize{10}{11}\selectfont ECON 3070-030}
\rfoot{\fontfamily{qbk}\fontsize{10}{11}\selectfont \thepage}


% Sections and Subsections

% define colors
\definecolor{buff-gold}{HTML}{CFB87C}
\definecolor{buff-grey}{HTML}{565A5C}
% custom tcolorbox
\tcbset{colframe=buff-gold, colback=white!100!black}

% new page per section
\usepackage{titlesec}
\newcommand{\sectionbreak}{\clearpage}
% change style of section
\usepackage{sectsty}
\sectionfont{\color{buff-gold} \fontfamily{qbk}\selectfont}
\subsectionfont{\color{buff-grey} \fontfamily{qbk}\selectfont}


\begin{document}
  
\section*{Problem Set 2}

\begin{enumerate}
  \item Suppose that Sam can buy only two goods with her income, bread ($B$) and eggs ($E$). Sam wants to buy the combination of bread and eggs that maximizes her utility, which is given by the following function:
  $$ 
    U(B,E) = 20B - \frac{1}{2} B^2 + 40E - \frac{1}{2} E^2
  $$

  \begin{enumerate}[(a)]
    \item Find Sam's marginal utility functions for both bread and eggs. Is Sam's marginal utility for eggs increasing, decreasing, or constant?
    

    \vspace*{80mm}
    \item Given her marginal utility functions, write the optimality condition that Sam's consumption of bread and eggs must satisfy. Explain why this condition must be satisfied in order for Sam's utility to be maximized
  
  \end{enumerate}

  
\newpage

\begin{enumerate}
  \item[] 
  \begin{enumerate}[(a)] 
    \item[(c)] Let the price of bread be $P_B$, the price of eggs be $P_E$, and Sam's income be $I$.Write Sam's budget constraint

    \vspace*{90mm}
    \item[(d)] Using Sam's budget constraint, and her optimality condition, solve for Sam's demand curves for bread and eggs, in terms of $P_B$, $P_E$, and $I$. You should find that your answer matches the demand curves below. (Note: In this question you will be graded on the steps you took to find the answer, so show all steps.)
    
    \newpage
    \item[(e)] Suppose that the price of bread, $P_B$, is $\$1$; the price of eggs, $P_E$, is $\$2$; and Sam's income is $\$40$. How much bread and eggs should Sam consume in order to maximize her utility?
    
    
    \vspace*{50mm}
    \item[(f)] Now, suppose that the price of bread, $P_B$ has changed to $\$2$. The price of eggs, $P_E$, is still $\$2$ and Sam's income is still $\$40$. Given the new price of bread, how much bread and eggs should Sam consume in order to maximize her utility?
    
    \vspace*{50mm}
    \item[(g)] Explain why Sam's consumption of bread and eggs changed in the way that it did
  \end{enumerate}
\end{enumerate}

\newpage
  \item For each of the following utility functions answer the following questions.
  	  \begin{enumerate}[(i)]
    \item Write the consumer's optimization problem
    \item Solve using the method of your choice for $x^{*}$ and $y^{*}$
    \item Solve for the demand functions, $x^*(P_x, P_y, I)$ and $y^*(P_x, P_y, I)$
  \end{enumerate}

  \begin{enumerate}[(a)]
    \item $U(x,y) = 2x + 4y \text{ and } P_x = 1, P_y = 3, I = 18$    
    
    \vspace*{90mm}
	\item $U(x,y) = \min(x, 2y) \text{ and } P_x = 2, P_y = 2, I = 24$
    
\newpage
    \item $U(x,y) = x^{\frac{1}{2}} y^{\frac{1}{2}}  \text{ and } P_x = 4, P_y = 4, I = 24$


	\vspace*{100mm}
	\item $U(x,y) = xy + y \text{ and } P_x = 2, P_y = 4, I = 60$    
  \end{enumerate}	  
\newpage
  \item Consider the utility function $U(x,y) = x^{1/3} y^{2/3}$. 
  
  \begin{enumerate}[(a)]
    \item Solve for the demand functions, $x^*(P_x, P_y, I)$ and $y^*(P_x, P_y, I)$
    
    % $$
    %   \frac{\frac{1}{3} x^{-2/3} y^{2/3}}{P_x} = \frac{\frac{2}{3} x^{1/3} y^{-1/3}}{P_y} \implies y = \frac{2 P_x}{P_y} x
    % $$
    % \begin{align*}
    %   I = P_x x + P_y y 
    %     &\implies I = P_x x + P_y \frac{2P_x}{P_y} x 
    %     = 3 P_x x \\
    %   &\implies x^*(P_x, P_y, I) = \frac{I}{3 P_x}
    % \end{align*}
    % Similarly, $y^*(P_x, P_y, I) = \frac{2 I}{3 P_y}$.


    \item At the prices $P_x = 1, P_y = 1$, sketch the engel curve for $x$ with $I = 3, 6,$ and $9$
    
  \end{enumerate}
  
  \newpage 
  \item Suppose that nearby Greeley, CO, following the example of Boulder, decides to impose it's own sugary drink tax. They agree on a tax of 4 cents per ounce. Also suppose that demand for soda (a type of sugary drink) in Greeley can be expressed by the following demand function:
  $$
    Q(P) = 20,000 - 500P
  $$
  where $P$ is the price of soda in \textbf{cents per ounce}.

  \begin{enumerate}[(a)]
    \item Plot the demand curve for soda in Greeley, with $Q$ on the horizontal axis and $P$ on the vertical axis. Graphically show the area of the graph that represents consumer surplus \textbf{before} the tax is imposed for a price of 10 cents per ounce. What is the total consumer surplus (\textbf{in dollars}) for the soda market in Greeley?
    
    \vspace*{70mm}
    \item Now suppose that the sugary drink tax takes effect. Assume that demand in Greeley will not impact the equilibrium market price for sugary drinks. In other words, Greeley residents must now pay a total price of 14 cents per ounce of soda (including the tax). What is the total consumer surplus (\textbf{in dollars}) for the soda market in Greeley \textbf{after} the tax takes effect?
    
    \vspace*{50mm}
    \item How much consumer surplus (\textbf{in dollars}) is lost due to the imposition of the tax?
  \end{enumerate}
  \newpage
\end{enumerate}

\end{document}