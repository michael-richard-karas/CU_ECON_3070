\documentclass[12pt,t]{beamer}
\input{preamble_lectures.tex}

\author{Michael R. Karas}
\title{Lecture 1 - Introduction}
\subtitle{ECON 3070 - Intermediate Microeconomic Theory}
\date{January X, 2025}

\begin{document}

% ------------------------------------------------------------------------------------------------
\begin{frame}
  \titlepage
\end{frame}
% ------------------------------------------------------------------------------------------------

% ------------------------------------------------------------------------------------------------
\begin{frame}{Why Study Microeconomics?}
  \begin{center}
    Resources are \textbf{scarce}, but \textbf{wants} are unlimited.
  \end{center}

  \pause\bigskip
  \begin{center}
    Economics tries to answer how people balance \bgPurple{trade-offs}
  \end{center}

  \pause\bigskip
  \begin{center}
    People respond to \bgAlice{incentives}
  \end{center}

  \pause\bigskip
  \begin{center}
    People make decisions \bgCoral{on the margin}
  \end{center}
\end{frame}
% ------------------------------------------------------------------------------------------------

% ------------------------------------------------------------------------------------------------
\begin{frame}{Why Study Microeconomics?}
  \bgPurple{Trade-offs}
  \begin{itemize}
    \item How do you chose to spend your (scarce) time? 
    \item How do you spend your (scarce) money?
  \end{itemize}
  
  \pause\bigskip
  \bgAlice{Incentives}
  \begin{itemize}
    \item How will people respond to a gas tax? (rebound-effect) 
    \item Why are B-cycles always broken? (tragedy of the commons)
  \end{itemize}
  
  \pause\bigskip
  \bgCoral{On the margin}
  \begin{itemize}
    \item Do I watch one more episode or go to bed? 
    \item How much pollution should we allow?
  \end{itemize}
\end{frame}
% ------------------------------------------------------------------------------------------------

% ------------------------------------------------------------------------------------------------
\begin{frame}{Analytical Tools}
  Any model requires us to specify which variables will taken as given and which will be determined by the model.
  \begin{itemize}
    \item An \navy{exogenous} variable is a variable that is taken as given in the model.
    \item An \navy{endogenous} variable is a variable that is determined by the model.
  \end{itemize}
\end{frame}
% ------------------------------------------------------------------------------------------------

% ------------------------------------------------------------------------------------------------
\begin{frame}{Analytical Tools}
  \bgCranberry{Try It Yourself}
  \bigskip
  
  Suppose you create a model to tell you what your estimated grade will be, based on how much you study, and how much you sleep the night before.
  \pause
  \begin{itemize}
    \item The amount of studying and amount of sleep are \navy{exogenous} variables.
    \item Your estimated grade is an \navy{endogenous} variable.
  \end{itemize}
\end{frame}
% ------------------------------------------------------------------------------------------------

% ------------------------------------------------------------------------------------------------
\begin{frame}{Analytical Tools}
  \bgCranberry{Try It Yourself}
  \bigskip
  
  Suppose you create a model to tell you how much money you will earn after you graduate, depending on your college major and on the estimated unemployment rate on the day that you graduate.
  \pause
  \begin{itemize}
    \item Your college major and unemployment rate are \navy{exogenous} variables.
    \item Your post-graduate earnings are an \navy{endogenous} variable.
  \end{itemize}
\end{frame}
% ------------------------------------------------------------------------------------------------

% ------------------------------------------------------------------------------------------------
\begin{frame}{Analytical Tools}
  Economic models: A formal description of a problem being addressed.

  \bigskip
  Examples:
  \begin{itemize}
    \item How the drought in California might affect the price of coffee in the United States.
    \item How an individual makes the decision of whether to attend college, vocational school, or neither.
    \item How a man or woman chooses their spouse/partner.
  \end{itemize}
\end{frame}
% ------------------------------------------------------------------------------------------------

% ------------------------------------------------------------------------------------------------
\begin{frame}{Three Tools of Microeconomic Analysis}
  Nearly all microeconomic models rely on just three key analytical tools:
  \begin{enumerate}
    \item Constrained optimization
    \item Equilibrium analysis
    \item Comparative statics
  \end{enumerate}

  \bigskip
  Throughout this course, we will use these three tools to analyze microeconomic problems.
\end{frame}
% ------------------------------------------------------------------------------------------------

% ------------------------------------------------------------------------------------------------
\begin{frame}{Calculus Aside}
  This class is going to require you to do a lot of algebra and take a lot of partial derivatives. People in this class are coming in with different skill levels, which is okay!

  \bigskip
  The next lecture will be a calculus review and you'll review problems in your first recitation.  So you will have time to practice.  If you don't spend the time now remembering how to take them, it will be difficult to succeed in the course.
\end{frame}
% ------------------------------------------------------------------------------------------------

% ------------------------------------------------------------------------------------------------
\begin{frame}{Constrained optimization}
  \textbf{Constrained optimization} is used when a decision maker seeks to make the optimal choice, taking into account any possible limitations or restrictions on choices

  \bigskip
  \bgPurple{Example} 
  
  \bigskip
  An individual seeks to maximize, through their choice of Friday night activities, their happiness.

  \begin{itemize}
    \item An \textbf{objective function} is the relationship that the decision maker seeks to optimize (e.g. maximize their Friday night).
    \item  The decision maker has to take into consideration their \textbf{constraints} (e.g. the individual only has \$20 to spend).
  \end{itemize}
\end{frame}
% ------------------------------------------------------------------------------------------------

% ------------------------------------------------------------------------------------------------
\begin{frame}{Constrained Optimization}
  A farmer needs to build a rectangular fence for her sheep. She has \textit{F} feet of fence, and is able to choose the dimensions (\textit{L} and \textit{W}) of her pen. Her goal is to maximize the area of the pen.

  \bigskip
  What is the objective function and what is the constraint?

  \pause
  \begin{itemize}
    \item The farmer's \textbf{objective function} is the function for the area of the pen, \textit{L*W}.
    \item The farmer's \textbf{constraint} is that they only have \textit{F} feet of fence.
  \end{itemize}
\end{frame}
% ------------------------------------------------------------------------------------------------

% ------------------------------------------------------------------------------------------------
\begin{frame}{Constrained Optimization}
  \bgCranberry{Try It Yourself} 

  \bigskip
  A baker has 8 hours in a day, during which he can bake cakes and brownies. If he bakes a cake, he can sell it for \$10, and it will take him 2 hours. If he bakes a tray of brownies, he can sell it for \$8, and it takes him 1 hour. However, the baker only has the supplies to bake at most 4 trays of brownies and 3 cakes in a given day. 
  
  \medskip
  How many brownies and cakes should the baker bake?
\end{frame}
% ------------------------------------------------------------------------------------------------

% ------------------------------------------------------------------------------------------------
\begin{frame}{Marginal Reasoning and Constrained Optimization}
  Continuing our example of the baker maximizing their profit. Suppose the baker is currently baking 3 cakes and 2 trays of brownies. Their profit is \$46.

  \bigskip
  What would his profit be if he instead baked 2 cakes, and 4 trays of brownies? 

  \begin{itemize}
    \item He would gain \$16 from the two additional brownie trays and lose \$8 for the cake.
    \item The additional profit gained, \$8 reflects the \textit{marginal} impact of the decision.
  \end{itemize}
\end{frame}
% ------------------------------------------------------------------------------------------------

% ------------------------------------------------------------------------------------------------
\begin{frame}{Margin = ``Of Another''}
  In economics, you will \emph{always} hear the world `marginal'. When you hear it, you can always replace it with the phrase `of another'

  \begin{itemize}
    \item So for example, if I say ``what is the marginal benefit of studying?'', I mean ``what is the benefit of studying another hour?''
  \end{itemize}
\end{frame}
% ------------------------------------------------------------------------------------------------

% ------------------------------------------------------------------------------------------------
\begin{frame}{Equilibrium Analysis}
  An \textbf{equilibrium} in a system is a state or condition that will continue indefinitely as long as exogenous factors remain unchanged.

  \bigskip
  In a competitive market, equilibrium occurs when the price is such that the quantity supplied by producers is equal to the quantity demanded by consumers.
\end{frame}
% ------------------------------------------------------------------------------------------------

% ------------------------------------------------------------------------------------------------
\begin{frame}[c]
  \begin{tikzpicture}
    \begin{axis}[
      width = 12cm,
      height = 9cm,
      xmin = 0, xmax = 30,
      ymin = 0, ymax = 9,
      axis lines = left,
      xtick = \empty, ytick = \empty,
      x label style={at={(axis description cs:0.5,-0.07)},anchor=north},
      y label style={at={(axis description cs:-0.07,.5)},anchor=south},
      xlabel = {\small Quantity (billion of bushels per year)},
      ylabel = {\small Price (dollars per bushel)},
      clip = false,
    ]
      % Demand Curve
      \addplot[color = alice, very thick] 
        coordinates {(0, 23/3) (11, 4) (14, 3) (23, 0)};
      \node [anchor = south west, text width = 3cm] at (23, 0) 
        {\small Demand};
        
      % Supply Curve
      \addplot[color = ruby, very thick] 
      coordinates {(4, 1/2) (9, 3) (11, 4) (15, 6)};
      \node [anchor = south west, text width = 3cm] at (15, 5.5) 
        {\small Supply};
    
      % Dotted lines
      \addplot[color = black, dotted, thick] 
        coordinates {(0, 4) (11, 4) (11, 0)};
      \addplot[color = black, dotted, thick] 
        coordinates {(0, 3) (9, 3) (9, 0)};
      \addplot[color = black, dotted, thick] 
        coordinates {(14, 3) (14, 0)};      

      % Curly brace
      \draw [decorate,decoration={brace,amplitude=2pt,mirror}, yshift = -0.2cm]
        (axis cs:9,3) -- (axis cs:14,3) node [black, midway, yshift = -0.8cm] 
        {\small 
          \begin{tabular}{l}Excess demand \\ when price is $\$3$\end{tabular}
        };
  
      % Coordinate Points
      \addplot[color = black, mark = *, only marks, mark size = 2pt] 
        coordinates {(9, 3) (11, 4) (14, 3)};
      \node [above] at (11, 4.25) {\small $E$};
  
      % Labels
      \node [left] at (0, 3) {\small $\$3$};
      \node [left] at (0, 4) {\small $\$4$};
      \node [below] at (0, 0) {\small $0$};
      \node [below] at (9, 0) {\small $9$};
      \node [below] at (11, 0) {\small $11$};
      \node [below] at (14, 0) {\small $14$};
    \end{axis}
  \end{tikzpicture}
\end{frame}
% ------------------------------------------------------------------------------------------------

% ------------------------------------------------------------------------------------------------
\begin{frame}[c]
  \begin{tikzpicture}
    \begin{axis}[
    width = 12cm,
    height = 9cm,
    xmin = 0, xmax = 30,
    ymin = 0, ymax = 9,
    axis lines = left,
    xtick = \empty, ytick = \empty,
    x label style={at={(axis description cs:0.5,-0.07)},anchor=north},
    y label style={at={(axis description cs:-0.07,.5)},anchor=south},
    xlabel = {\small Quantity (billion of bushels per year)},
    ylabel = {\small Price (dollars per bushel)},
    clip = false,
    ]
      % Demand Curve
      \addplot[color = alice, very thick] 
        coordinates {(0, 23/3) (11, 4) (14, 3) (23, 0)};
      \node [anchor = south west, text width = 3cm] at (23, 0) 
        {\small Demand};
        
      % Supply Curve
      \addplot[color = ruby, very thick] 
      coordinates {(4, 1/2) (9, 3) (11, 4) (15, 6)};
      \node [anchor = south west, text width = 3cm] at (15, 5.5) 
        {\small Supply};
    
      % Dotted lines
      \addplot[color = black, dotted, thick] 
        coordinates {(0, 4) (11, 4) (11, 0)};
      \addplot[color = black, dotted, thick] 
        coordinates {(0, 5) (8, 5) (8, 0)};
      \addplot[color = black, dotted, thick] 
        coordinates {(13, 5) (13, 0)};      

      % Curly brace
      \draw [decorate,decoration={brace,amplitude=2pt}, yshift = 0.2cm]
        (axis cs:8,5) -- (axis cs:13,5) node [black, midway, yshift = 0.8cm] 
        {\small \begin{tabular}{l}Excess supply \\ when price is $\$5$\end{tabular} };
  
      % Coordinate Points
      \addplot[color = black, mark = *, only marks, mark size = 2pt] 
        coordinates {(8, 5) (11, 4) (13, 5)};
      \node [above] at (11, 4.25) {\small $E$};
  
      % Labels
      \node [left] at (0, 3) {\small $\$3$};
      \node [left] at (0, 4) {\small $\$4$};
      \node [below] at (0, 0) {\small $0$};
      \node [below] at (8, 0) {\small $8$};
      \node [below] at (11, 0) {\small $11$};
      \node [below] at (13, 0) {\small $13$};
    \end{axis}
  \end{tikzpicture}
\end{frame}
% ------------------------------------------------------------------------------------------------

% ------------------------------------------------------------------------------------------------
\begin{frame}{Comparative Statics}
  \textbf{Comparative statics} analysis is used to examine how a shock to a system will affect another variable in an economic model. 
  
  \begin{itemize}
    \item This is really useful since it let's you decide whether or not to take an action by predicting how it affects variables you care about
  \end{itemize}

  Comparative statics analysis can be applied to constrained optimization or to equilibrium analysis.
\end{frame}
% ------------------------------------------------------------------------------------------------

% ------------------------------------------------------------------------------------------------
\begin{frame}{Comparative Statics}
  Consider the market for pistachios:

  \begin{itemize}
    \item We can see what will happen to the price and quantity of pistachios in the market when the supply curve shifts.
    \item The shift in the supply curve represents an exogenous shock.
  \end{itemize}
\end{frame}
% ------------------------------------------------------------------------------------------------

% ------------------------------------------------------------------------------------------------
\begin{frame}[c]
  \begin{tikzpicture}
    \begin{axis}[
    width = 12cm,
    height = 9cm,
    xmin = 0, xmax = 25,
    ymin = 0, ymax = 9,
    axis lines = left,
    xtick = \empty, ytick = \empty,
    x label style={at={(axis description cs:1,0)},anchor=north east},
    y label style={at={(axis description cs:0,1)},anchor=south east},
    xlabel = {Quantity},
    ylabel = {Price},
    clip = false,
    ]
      % Demand Curve
      \addplot[color = alice, very thick] 
        coordinates {(0, 23/3) (11, 4) (14, 3) (23, 0)};
      \node [anchor = south west, text width = 3cm] at (23, 0) 
        {\small $D$};
        
      % Supply Curve 1
      \addplot[color = ruby!50!white, very thick] 
      coordinates {(4, 1/2) (9, 3) (11, 4) (15, 6)};
      \node [anchor = south west, text width = 3cm] at (15, 5.5) 
        {\small $S_1$};

      % Supply Curve 2
      \addplot[color = ruby, very thick] 
      coordinates {(0, 1/2) (5, 3) (7, 4) (11, 6) (15, 8)};
      \node [anchor = south west, text width = 3cm] at (15, 7.5) 
        {\small $S_2$};
     

      % Coordinate Points
      \addplot[color = black, mark = *, only marks, mark size = 2pt] 
        coordinates {(8.6, 4.8) (11, 4)};
      \node [above] at (11, 4.25) {\small $A$};
      \node [above] at (8.5, 5) {\small $B$};

      % Curve shift arrow
      \draw[-stealth, very thick, ruby] (axis cs:13.8, 5.5) -- (axis cs:10.2, 5.5);
  
      % Dotted lines
      \addplot[color = black, dotted, thick] 
        coordinates {(0, 4) (11, 4) (11, 0)};
      \addplot[color = black, dotted, thick] 
        coordinates {(0, 4.8) (8.5, 4.8) (8.5, 0)};

      % Axis Labels
      \node [anchor = south, rotate = 90] at (axis cs: 0, 4.4) 
      {\begin{tabular}{c}{\small Equilibrium price goes up} \\ $\rightarrow$\end{tabular}};

      \node [anchor = north] at (axis cs: 9.8, 0) 
      {\begin{tabular}{c}$\leftarrow$ \\ {\small Equilibrium quantity goes down}\end{tabular}};

    \end{axis}
  \end{tikzpicture}
\end{frame}
% ------------------------------------------------------------------------------------------------

% ------------------------------------------------------------------------------------------------
\begin{frame}{Positive and Normative Analysis}
  \textbf{Positive analysis} attempts to explain how an economic system works or to predict how it will change over time.

  \begin{itemize}
    \item Explanatory questions, such as "What has happened?"
  \end{itemize}

  \bigskip
  \textbf{Normative analysis} asks prescriptive questions.

  \begin{itemize}
    \item Involves value judgments.
  \end{itemize}

  \bigskip
  \begin{center}
  This class will focus on positive analysis.
  \end{center}
\end{frame}
% ------------------------------------------------------------------------------------------------

% ------------------------------------------------------------------------------------------------
\begin{frame}{Positive and Normative Analysis}
  
  Which of the following is a positive statement?

  \bigskip
  \begin{enumerate}[A)]
    \item Rent controls (a rent ceiling) will cause a housing shortage.
    \item It would be good for the U.S if we stop trading with China.
    \item The minimum wage should be raised to make poor people better off.
    \item Businesses need to pay their employees a living wage.
  \end{enumerate}
\end{frame}

\end{document}
% ------------------------------------------------------------------------------------------------

\end{document}