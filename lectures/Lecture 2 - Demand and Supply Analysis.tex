\documentclass[12pt,t]{beamer}
% xcolor and define colors -------------------------
\usepackage{xcolor}

% https://www.viget.com/articles/color-contrast/
\definecolor{purple}{HTML}{695693}
\definecolor{navy}{HTML}{567293}
\definecolor{ruby}{HTML}{9a2515}
\definecolor{alice}{HTML}{107895}
\definecolor{daisy}{HTML}{EBC944}
\definecolor{coral}{HTML}{F26D21}
\definecolor{kelly}{HTML}{829356}
\definecolor{cranberry}{HTML}{E64173}
\definecolor{jet}{HTML}{131516}
\definecolor{asher}{HTML}{555F61}
\definecolor{slate}{HTML}{314F4F}

% Main theme colors
\definecolor{accent}{HTML}{107895}
\definecolor{accent2}{HTML}{9a2515}

\newcommand\navy[1]{{\color{navy}#1}}
\newcommand\purple[1]{{\color{purple}#1}}
\newcommand\kelly[1]{{\color{kelly}#1}}
\newcommand\ruby[1]{{\color{ruby}#1}}
\newcommand\alice[1]{{\color{alice}#1}}
\newcommand\daisy[1]{{\color{daisy}#1}}
\newcommand\coral[1]{{\color{coral}#1}}
\newcommand\cranberry[1]{{\color{cranberry}#1}}
\newcommand\slate[1]{{\color{slate}#1}}
\newcommand\jet[1]{{\color{jet}#1}}
\newcommand\asher[1]{{\color{asher}#1}}

\newcommand\bgNavy[1]{{\colorbox{navy!80!white}{\textcolor{white}{#1}}}}
\newcommand\bgPurple[1]{{\colorbox{purple!80!white}{\textcolor{white}{#1}}}}
\newcommand\bgKelly[1]{{\colorbox{kelly!80!white}{\textcolor{white}{#1}}}}
\newcommand\bgRuby[1]{{\colorbox{ruby!80!white}{\textcolor{white}{#1}}}}
\newcommand\bgAlice[1]{{\colorbox{alice!80!white}{\textcolor{white}{#1}}}}
\newcommand\bgDaisy[1]{{\colorbox{daisy!80!white}{\textcolor{white}{#1}}}}
\newcommand\bgCoral[1]{{\colorbox{coral!80!white}{\textcolor{white}{#1}}}}
\newcommand\bgCranberry[1]{{\colorbox{cranberry!80!white}{\textcolor{white}{#1}}}}


% Beamer Options -------------------------------------

% Background
\setbeamercolor{background canvas}{bg = white}

% Change text margins
\setbeamersize{text margin left = 15pt, text margin right = 15pt} 

% \alert
\setbeamercolor{alerted text}{fg = accent2}

% Frame title
\setbeamercolor{frametitle}{bg = white, fg = jet}
\setbeamercolor{framesubtitle}{bg = white, fg = accent}
\setbeamerfont{framesubtitle}{size = \small, shape = \itshape}

% Block
\setbeamercolor{block title}{fg = white, bg = accent2}
\setbeamercolor{block body}{fg = jet, bg = jet!10!white}

% Title page
\setbeamercolor{title}{fg = jet}
\setbeamercolor{subtitle}{fg = accent}

%% Custom \maketitle and \titlepage
\setbeamertemplate{title page}
{
    %\begin{centering}
        \vspace{20mm}
        {\Large \usebeamerfont{title}\usebeamercolor[fg]{title}\inserttitle}\\ \vskip0.25em%
        \ifx\insertsubtitle\@empty%
        \else%
          {\usebeamerfont{subtitle}\usebeamercolor[fg]{subtitle}\insertsubtitle\par}%
        \fi% 
        {\vspace{10mm}\insertauthor}\\
        {\color{asher}\small{\insertdate}}\\
    %\end{centering}
}

% Table of Contents
\setbeamercolor{section in toc}{fg = accent!70!jet}
\setbeamercolor{subsection in toc}{fg = jet}

% Button 
\setbeamercolor{button}{bg = accent}

% Remove navigation symbols
\setbeamertemplate{navigation symbols}{}

% Optional: page numbers at bottom
\addtobeamertemplate{navigation symbols}{}{%
    \usebeamerfont{footline}%
    \hspace{1em}%
    \alice{\insertframenumber/\inserttotalframenumber}
    \vspace*{1.5mm}
}


% Table and Figure captions
\setbeamercolor{caption}{fg=jet!70!white}
\setbeamercolor{caption name}{fg=jet}
\setbeamerfont{caption name}{shape = \itshape}

% Bullet points

%% Fix left-margins
\settowidth{\leftmargini}{\usebeamertemplate{itemize item}}
\addtolength{\leftmargini}{\labelsep}

%% enumerate item color
\setbeamercolor{enumerate item}{fg = accent}
\setbeamerfont{enumerate item}{size = \small}
\setbeamertemplate{enumerate item}{\insertenumlabel.}

%% enumerate subitem color
\setbeamercolor{enumerate subitem}{fg = accent!60!white}
\setbeamerfont{enumerate subitem}{size = \small}
\setbeamertemplate{enumerate subitem}{\insertenumlabel.}

%% itemize
\setbeamercolor{itemize item}{fg = accent!70!white}
\setbeamerfont{itemize item}{size = \small}
\setbeamertemplate{itemize item}[circle]

%% right arrow for subitems
\setbeamercolor{itemize subitem}{fg = accent!60!white}
\setbeamerfont{itemize subitem}{size = \small}
\setbeamertemplate{itemize subitem}{$\rightarrow$}

\setbeamertemplate{itemize subsubitem}[square]
\setbeamercolor{itemize subsubitem}{fg = jet}
\setbeamerfont{itemize subsubitem}{size = \small}

% References

%% Bibliography Font, roughly matching aea
\setbeamerfont{bibliography item}{size = \footnotesize}
\setbeamerfont{bibliography entry author}{size = \footnotesize, series = \bfseries}
\setbeamerfont{bibliography entry title}{size = \footnotesize}
\setbeamerfont{bibliography entry location}{size = \footnotesize, shape = \itshape}
\setbeamerfont{bibliography entry note}{size = \footnotesize}

\setbeamercolor{bibliography item}{fg = jet}
\setbeamercolor{bibliography entry author}{fg = accent!60!jet}
\setbeamercolor{bibliography entry title}{fg = jet}
\setbeamercolor{bibliography entry location}{fg = jet}
\setbeamercolor{bibliography entry note}{fg = jet}

%% Remove bibliography symbol in slides
\setbeamertemplate{bibliography item}{}





% Links ----------------------------------------------

\usepackage{hyperref}
\hypersetup{
  colorlinks = true,
  linkcolor = accent2,
  filecolor = accent2,
  urlcolor = accent2,
  citecolor = accent2,
}


% Line spacing --------------------------------------
\usepackage{setspace}
% \setdisplayskipstretch{2}
\setstretch{1.3}


% \begin{columns} -----------------------------------
\usepackage{multicol}


% Fonts ---------------------------------------------
% Beamer Option to use custom fonts
\usefonttheme{professionalfonts}

% \usepackage[utopia, smallerops, varg]{newtxmath}
% \usepackage{utopia}
\usepackage[sfdefault,light]{roboto}

% Small adjustments to text kerning
\usepackage{microtype}



% Remove annoying over-full box warnings -----------
\vfuzz2pt 
\hfuzz2pt


% Table of Contents with Sections
\setbeamerfont{myTOC}{series=\bfseries, size=\Large}
\AtBeginSection[]{
        \frame{
            \frametitle{Roadmap}
            \tableofcontents[current]   
        }
    }


% References ----------------------------------------
\usepackage[
    citestyle= authoryear,
    style = authoryear,
    natbib = true, 
    backend = biber
]{biblatex}

% Smaller font-size for references
\renewcommand*{\bibfont}{\small}

% Remove "In:"
\renewbibmacro{in:}{}

% Color citations for slides
\newenvironment{citecolor}
    {\footnotesize\begin{color}{accent2}}
    {\end{color}}

\newcommand{\citetcolor}[1]{{\footnotesize\textcolor{gray}{\citet{#1}}}}
\newcommand{\citepcolor}[1]{{\footnotesize\textcolor{gray}{\citep{#1}}}}

% Tables -------------------------------------------
% Tables too big
% \begin{adjustbox}{width = 1.2\textwidth, center}
\usepackage{adjustbox}
\usepackage{array}
\usepackage{threeparttable, booktabs, adjustbox}
    
% Fix \input with tables
% \input fails when \\ is at end of external .tex file

\makeatletter
\let\input\@@input
\makeatother

% Tables too narrow
% \begin{tabularx}{\linewidth}{cols}
% col-types: X - center, L - left, R -right
% Relative scale: >{\hsize=.8\hsize}X/L/R
\usepackage{tabularx}
\newcolumntype{L}{>{\raggedright\arraybackslash}X}
\newcolumntype{R}{>{\raggedleft\arraybackslash}X}
\newcolumntype{C}{>{\centering\arraybackslash}X}

% Figures

% \imageframe{img_name} -----------------------------
% from https://github.com/mattjetwell/cousteau
\newcommand{\imageframe}[1]{%
    \begin{frame}[plain]
        \begin{tikzpicture}[remember picture, overlay]
            \node[at = (current page.center), xshift = 0cm] (cover) {%
                \includegraphics[keepaspectratio, width=\paperwidth, height=\paperheight]{#1}
            };
        \end{tikzpicture}
    \end{frame}%
}

% subfigures
\usepackage{subfigure}

% Strikeout text
\usepackage{cancel}

% Highlight slide -----------------------------------
% \begin{transitionframe} Text \end{transitionframe}
% from paulgp's beamer tips
\newenvironment{transitionframe}{
    \setbeamercolor{background canvas}{bg=accent!60!black}
    \begin{frame}\color{accent!10!white}\LARGE\centering
}{
    \end{frame}
}


% Table Highlighting --------------------------------
% Create top-left and bottom-right markets in tabular cells with a unique matching id and these commands will outline those cells
\usepackage[beamer,customcolors]{hf-tikz}
\usetikzlibrary{calc,fit,shapes.misc,backgrounds}
\usepackage{pgfplots}
\pgfplotsset{compat = newest}
\usetikzlibrary{positioning, arrows.meta}
\usepgfplotslibrary{fillbetween}

% halo around text
%https://tex.stackexchange.com/questions/18472/tikz-halo-around-text
\usepackage[outline]{contour} 
\contourlength{1.2pt}
\tikzset{
  contour text/.style={node contents={\contour{white}{#1}}},
  halo text node/.style={circle, draw, pattern=north east lines}
}


\def\arraystretch{0.75}

% To set the hypothesis highlighting boxes red.
\newcommand\marktopleft[1]{%
    \tikz[overlay,remember picture] 
        \node (marker-#1-a) at (0,1.5ex) {};%
}
\newcommand\markbottomright[1]{%
    \tikz[overlay,remember picture] 
        \node (marker-#1-b) at (0,0) {};%
    \tikz[accent!80!jet, ultra thick, overlay, remember picture, inner sep=4pt]
        \node[draw, rectangle, fit=(marker-#1-a.center) (marker-#1-b.center)] {};%
}


\author{Michael R. Karas}
\title{Lecture 2 - Demand and Supply}
\subtitle{ECON 3070 - Intermediate Microeconomic Theory}
\date{January X, 2025}

\begin{document}

% ------------------------------------------------------------------------------------------------
\begin{frame}
  \titlepage
\end{frame}
% ------------------------------------------------------------------------------------------------

% ------------------------------------------------------------------------------------------------
\begin{frame}{Overview}
  In this lecture, we will review the following:
  \begin{enumerate}
    \item Supply and demand

    \item Market equilibrium, both graphically and numerically

    \item Elasticities
  \end{enumerate}
\end{frame}
% ------------------------------------------------------------------------------------------------

% ------------------------------------------------------------------------------------------------
\begin{frame}{Demand, Supply and Market Equilibrium}
  A \textit{perfectly competitive market} comprises a large number of buyers and sellers.
  \begin{itemize}
    \item Buyers and sellers act as \textit{price takers} in these markets.
    \item Sellers all produce identical products.
  \end{itemize}
\end{frame}
% ------------------------------------------------------------------------------------------------

% ------------------------------------------------------------------------------------------------
\begin{frame}[c]
  \begin{tikzpicture}
    \begin{axis}[
    width = 12cm,
    height = 9cm,
    xmin = 0, xmax = 30,
    ymin = 0, ymax = 9,
    axis lines = left,
    xtick = \empty, ytick = \empty,
    x label style={at={(axis description cs:0.5,-0.07)},anchor=north},
    y label style={at={(axis description cs:-0.07,.5)},anchor=south},
    xlabel = {\small Quantity (billion of bushels per year)},
    ylabel = {\small Price (dollars per bushel)},
    clip = false,
    ]
      % Demand Curve
      \addplot[color = alice, very thick] 
        coordinates {(0, 23/3) (11, 4) (14, 3) (23, 0)};
      \node [anchor = south west, text width = 3cm] at (23, 0) 
        {\small Demand};
        
      % Supply Curve
      \addplot[color = ruby, very thick] 
      coordinates {(4, 1/2) (9, 3) (11, 4) (15, 6)};
      \node [anchor = south west, text width = 3cm] at (15, 5.5) 
        {\small Supply};
    
      % Dotted lines
      \addplot[color = black, dotted, thick] 
        coordinates {(0, 3) (14, 3) (14, 0)};
      \addplot[color = black, dotted, thick] 
        coordinates {(0, 4) (11, 4) (11, 0)};
      \addplot[color = black, dotted, thick] 
        coordinates {(9, 0) (9, 3)};
  
      % Coordinate Points
      \addplot[color = black, mark = *, only marks, mark size = 2pt] 
        coordinates {(9, 3) (11, 4) (14, 3)};
      \node [above] at (11, 4.25) {\small $E$};
  
      % Labels
      \node [left] at (0, 3) {\small $\$3$};
      \node [left] at (0, 4) {\small $\$4$};
      \node [below] at (0, 0) {\small $0$};
      \node [below] at (9, 0) {\small $9$};
      \node [below] at (11, 0) {\small $11$};
      \node [below] at (14, 0) {\small $14$};
    \end{axis}
  \end{tikzpicture}
\end{frame}
% ------------------------------------------------------------------------------------------------

% ------------------------------------------------------------------------------------------------
\begin{frame}{Demand, Supply and Market Equilibrium}
  \bgCranberry{Try It Yourself}

\bigskip
Which of the following would most likely be considered a perfectly competitive market?

\bigskip
  \begin{itemize}
    \item The craft beer industry
    \item The grill industry
    \item The soybean industry
  \end{itemize}
\end{frame}
% ------------------------------------------------------------------------------------------------

% ------------------------------------------------------------------------------------------------
\begin{frame}{Demand Curves}
  \textbf{The market demand curve} tells us the quantity of corn that buyers are willing to purchase at different prices.
  \begin{itemize}
    \item \textbf{Derived demand} is derived from the production and sale of other goods. (\emph{e.g.} computer chips are not purchased directly, but are used as an input for computers/phones.)
    \item \textbf{Direct demand} is demand that comes directly from consumers.
  \end{itemize}

  \pause\bigskip
  The \textbf{law of demand} is the inverse relationship between the price of a good and the quantity demanded of that good.
\end{frame}
% ------------------------------------------------------------------------------------------------

% ------------------------------------------------------------------------------------------------
\begin{frame}{Calculating Quantity Demanded}
  \bgCranberry{Try It Yourself}

\bigskip
Suppose that the demand curve for Chaco's sandals in Boulder is given by $Q^D_{chaco} = 40,000  - 500P_{chaco}$. What is the quantity demanded of Chaco's if the price is $ \$ 60 $?

\end{frame}
% ------------------------------------------------------------------------------------------------

% ------------------------------------------------------------------------------------------------
\begin{frame}{Aggregating Demand}
  Consider if you have two consumers of a good. Each consumer's demand curve tells us \emph{at a given price}, how many units will they buy.

  \bigskip
  How do we figure out the aggregate demand curve? 
  
  \pause 
  \begin{itemize}
    \item \emph{At a given price}, add up each consumer's quantity demanded
  \end{itemize}
\end{frame}

\begin{frame}{Aggregating Demand}
  Demand cannot be negative. So when we state demand as
  $$
    Q^D_{chaco} = 40,000  - 500P_{chaco}
  $$
  we are actually saying
  $$
    Q^D_{chaco} =
    \begin{cases}
      40,000  - 500P_{chaco} & \text{if } P_{chaco} \leq 80 \\
      0                      & \text{otherwise}.
    \end{cases}
  $$
\end{frame}
% ------------------------------------------------------------------------------------------------

% ------------------------------------------------------------------------------------------------
\begin{frame}{Aggregating Demand}{Example}
  Suppose we have two consumers, A and B. Suppose that
  $$
    Q_A^D =
    \begin{cases}
      20 - 2P & \text{if } P \leq 10 \\
      0       & \text{otherwise}.
    \end{cases}
    \quad\quad
    Q_B^D =
    \begin{cases}
      21 - 3P & \text{if } P  \leq 7 \\
      0       & \text{otherwise}.
    \end{cases}
  $$

  \bigskip
  \alice{What's the aggregate demand curve?}
\end{frame}
% ------------------------------------------------------------------------------------------------

% ------------------------------------------------------------------------------------------------
\begin{frame}{Aggregating Demand}{Example}
  $$
    Q_A^D =
    \begin{cases}
      20 - 2P & \text{if } P \leq 10 \\
      0       & \text{otherwise}.
    \end{cases}
    \quad\quad
    Q_B^D =
    \begin{cases}
      21 - 3P & \text{if } P  \leq 7 \\
      0       & \text{otherwise}.
    \end{cases}
  $$

  \vspace{10mm}
  \only<1>{
    At a price of $P \leq 7$, both consumers will purchase good, so aggregate demand is given by
    $$Q^D_{mkt} = Q_A^D + Q_B^D = (20 - 2P) + (21 - 3P) = 41 - 5P$$
  }

  \only<2>{
    At a price $7 < P \leq 10$, only consumer A will consume ($Q_B^D = 0$), so aggregate demand is given by
    $$Q^D_{mkt} = Q_A^D + Q_B^D = (20 - 2P) + 0 = 20 - 2P$$
  }

  \only<3> {
    At a price $P > 10$, no one consumes anything, so 
    $$Q^D_{mkt} = Q_A^D + Q_B^D = 0 + 0 = 0$$ 
  }
\end{frame}
% ------------------------------------------------------------------------------------------------

% ------------------------------------------------------------------------------------------------
\begin{frame}{Aggregating Demand}{Example}
  Putting this together
  $$
    Q_{mkt}^D =
    \begin{cases}
      41 - 5P & \text{if } P \leq 7      \\
      20 - 2P & \text{if } 7 < P \leq 10 \\
      0       & \text{otherwise}.
    \end{cases}
  $$
\end{frame}

\begin{frame}[c]
  \begin{tikzpicture}
    \begin{axis}[
    width = 12cm,
    height = 9cm,
    xmin = 0, xmax = 24,
    ymin = 0, ymax = 6,
    axis lines = left,
    xtick = \empty, ytick = \empty,
    x label style={at={(axis description cs:0.5,-0.07)},anchor=north},
    y label style={at={(axis description cs:-0.07,.5)},anchor=south},
    xlabel = {\small $Q$ (liters of orange juice per month)},
    ylabel = {\small $P$ (dollars per liter},
    clip = false,
    ]
      % Demand Curves
      \addplot[color = alice!50!white, very thick] 
        coordinates {(0, 3) (6, 0)};
      \node [anchor = south west, text width = 3cm] at (5.5, 0) 
        {\small $D_A$};
      
      \addplot[color = alice!50!white, very thick] 
        coordinates {(0, 5) (15, 0)};
      \node [anchor = south west, text width = 3cm] at (15, 0) 
        {\small $D_B$};

      % Aggregate Demand
      \addplot[color = alice, very thick] 
      coordinates {(0, 5) (6, 3) (21, 0)};
      \node [anchor = south west, text width = 3cm] at (21, 0) 
        {\small $D_{mkt}$};
        
    
      % Dotted lines
      \addplot[color = black, dotted, thick] 
        coordinates {(0, 1) (16, 1) (16, 0)};
      \addplot[color = black, dotted, thick] 
        coordinates {(4, 1) (4, 0)};
      \addplot[color = black, dotted, thick] 
        coordinates {(12, 1) (12, 0)};

      % Coordinate Points
      \node [left] at (0, 3) {\small $\$3$};
  
      % Labels
      \node [left] at (0, 0) {\small $0$};
      \node [left] at (0, 1) {\small $1$};
      \node [left] at (0, 2) {\small $2$};
      \node [left] at (0, 3) {\small $3$};
      \node [left] at (0, 4) {\small $4$};
      \node [left] at (0, 5) {\small $\$5$};
      \node [below] at (0, 0) {\small $0$};
      \node [below] at (4, 0) {\small $4$};
      \node [below] at (6, 0) {\small $6$};
      \node [below] at (12, 0) {\small $12$};
      \node [below] at (15, 0) {\small $15$};
      \node [below] at (16, 0) {\small $16$};
      \node [below] at (21, 0) {\small $21$};
    \end{axis}
  \end{tikzpicture}
\end{frame}
% ------------------------------------------------------------------------------------------------

% ------------------------------------------------------------------------------------------------
\begin{frame}[c]
  \begin{tikzpicture}
    \begin{axis}[
    width = 12cm,
    height = 9cm,
    xmin = 0, xmax = 24,
    ymin = 0, ymax = 6,
    axis lines = left,
    xtick = \empty, ytick = \empty,
    x label style={at={(axis description cs:0.5,-0.07)},anchor=north},
    y label style={at={(axis description cs:-0.07,.5)},anchor=south},
    xlabel = {\small $Q$ (liters of orange juice per month)},
    ylabel = {\small $P$ (dollars per liter},
    clip = false,
    ]
      % Demand Curves
      \addplot[color = alice!50!white, very thick] 
        coordinates {(0, 3) (6, 0)};
      \node [anchor = south west, text width = 3cm] at (5.5, 0) 
        {\small $D_A$};
      
      \addplot[color = alice!50!white, very thick] 
        coordinates {(0, 5) (15, 0)};
      \node [anchor = south west, text width = 3cm] at (15, 0) 
        {\small $D_B$};

      % Aggregate Demand
      \addplot[color = alice, very thick] 
      coordinates {(0, 5) (6, 3) (21, 0)};
      \node [anchor = south west, text width = 3cm] at (21, 0) 
        {\small $D_{mkt}$};
        
    
      % Dotted lines
      \addplot[color = black, dotted, thick] 
        coordinates {(0, 4) (3, 4) (3, 0)};

      % Coordinate Points
      \node [left] at (0, 3) {\small $\$3$};
  
      % Labels
      \node [left] at (0, 0) {\small $0$};
      \node [left] at (0, 1) {\small $1$};
      \node [left] at (0, 2) {\small $2$};
      \node [left] at (0, 3) {\small $3$};
      \node [left] at (0, 4) {\small $4$};
      \node [left] at (0, 5) {\small $\$5$};
      \node [below] at (0, 0) {\small $0$};
      \node [below] at (4, 0) {\small $4$};
      \node [below] at (6, 0) {\small $6$};
      \node [below] at (12, 0) {\small $12$};
      \node [below] at (15, 0) {\small $15$};
      \node [below] at (16, 0) {\small $16$};
      \node [below] at (21, 0) {\small $21$};
    \end{axis}
  \end{tikzpicture}
\end{frame}
% ------------------------------------------------------------------------------------------------

% ------------------------------------------------------------------------------------------------
\begin{frame}
  \bgCranberry{Try It Yourself}

  \bigskip
  Find the aggregate demand curve:
  $$
    Q_A^D =
    \begin{cases}
      30 - 6P & \text{if } P \leq 5 \\
      0       & \text{otherwise}.
    \end{cases}
    \quad\quad
    Q_B^D =
    \begin{cases}
      32 - 4P & \text{if } P  \leq 8 \\
      0       & \text{otherwise}.
    \end{cases}
  $$
\end{frame}
% ------------------------------------------------------------------------------------------------

% ------------------------------------------------------------------------------------------------
\begin{frame}{Supply Curves}
  \textbf{The market supply curve} tells us the quantity of a product that producers are willing to sell at different prices.

  \begin{itemize}
    \item The \textbf{law of supply} is the \emph{positive} relationship between the price of a good and the quantity supplied of that good.
  \end{itemize}

  \pause
  \bigskip
  Quantity supplied is affected by not just market price. For example, the prices of \textbf{factors of production}, or resources used in the production of the good, affect the quantity supplied.
\end{frame}
% ------------------------------------------------------------------------------------------------

% ------------------------------------------------------------------------------------------------
\begin{frame}{Calculating Quantity Supplied}
  \bgCranberry{Try It Yourself}

\bigskip
Suppose that the supply curve for Chaco's sandals in Boulder is given by $Q^S_{chaco} = -8,000 + 300P_{chaco}$. What is the quantity supplied of Chaco's if the price is $ \$ 50 $?

\end{frame}
% ------------------------------------------------------------------------------------------------

% ------------------------------------------------------------------------------------------------
\begin{frame}[c]
  \begin{tikzpicture}
    \begin{axis}[
      width = 12cm,
      height = 9cm,
      xmin = 0, xmax = 30,
      ymin = 0, ymax = 9,
      axis lines = left,
      xtick = \empty, ytick = \empty,
      x label style={at={(axis description cs:0.5,-0.07)},anchor=north},
      y label style={at={(axis description cs:-0.07,.5)},anchor=south},
      xlabel = {\small Quantity (billion of bushels per year)},
      ylabel = {\small Price (dollars per bushel)},
      clip = false,
    ]
      % Demand Curve
      \addplot[color = alice, very thick] 
        coordinates {(0, 23/3) (11, 4) (14, 3) (23, 0)};
      \node [anchor = south west, text width = 3cm] at (23, 0) 
        {\small Demand};
        
      % Supply Curve
      \addplot[color = ruby, very thick] 
      coordinates {(4, 1/2) (9, 3) (11, 4) (15, 6)};
      \node [anchor = south west, text width = 3cm] at (15, 5.5) 
        {\small Supply};
    
      % Dotted lines
      \addplot[color = black, dotted, thick] 
        coordinates {(0, 4) (11, 4) (11, 0)};
      \addplot[color = black, dotted, thick] 
        coordinates {(0, 3) (9, 3) (9, 0)};
      \addplot[color = black, dotted, thick] 
        coordinates {(14, 3) (14, 0)};      

      % Curly brace
      \draw [decorate,decoration={brace,amplitude=2pt,mirror}, yshift = -0.2cm]
        (axis cs:9,3) -- (axis cs:14,3) node [black, midway, yshift = -0.8cm] 
        {\small 
          \begin{tabular}{l}Excess demand \\ when price is $\$3$\end{tabular}
        };
  
      % Coordinate Points
      \addplot[color = black, mark = *, only marks, mark size = 2pt] 
        coordinates {(9, 3) (11, 4) (14, 3)};
      \node [above] at (11, 4.25) {\small $E$};
  
      % Labels
      \node [left] at (0, 3) {\small $\$3$};
      \node [left] at (0, 4) {\small $\$4$};
      \node [below] at (0, 0) {\small $0$};
      \node [below] at (9, 0) {\small $9$};
      \node [below] at (11, 0) {\small $11$};
      \node [below] at (14, 0) {\small $14$};
    \end{axis}
  \end{tikzpicture}
\end{frame}
% ------------------------------------------------------------------------------------------------

% ------------------------------------------------------------------------------------------------
\begin{frame}[c]
  \begin{tikzpicture}
    \begin{axis}[
    width = 12cm,
    height = 9cm,
    xmin = 0, xmax = 30,
    ymin = 0, ymax = 9,
    axis lines = left,
    xtick = \empty, ytick = \empty,
    x label style={at={(axis description cs:0.5,-0.07)},anchor=north},
    y label style={at={(axis description cs:-0.07,.5)},anchor=south},
    xlabel = {\small Quantity (billion of bushels per year)},
    ylabel = {\small Price (dollars per bushel)},
    clip = false,
    ]
      % Demand Curve
      \addplot[color = alice, very thick] 
        coordinates {(0, 23/3) (11, 4) (14, 3) (23, 0)};
      \node [anchor = south west, text width = 3cm] at (23, 0) 
        {\small Demand};
        
      % Supply Curve
      \addplot[color = ruby, very thick] 
      coordinates {(4, 1/2) (9, 3) (11, 4) (15, 6)};
      \node [anchor = south west, text width = 3cm] at (15, 5.5) 
        {\small Supply};
    
      % Dotted lines
      \addplot[color = black, dotted, thick] 
        coordinates {(0, 4) (11, 4) (11, 0)};
      \addplot[color = black, dotted, thick] 
        coordinates {(0, 5) (8, 5) (8, 0)};
      \addplot[color = black, dotted, thick] 
        coordinates {(13, 5) (13, 0)};      

      % Curly brace
      \draw [decorate,decoration={brace,amplitude=2pt}, yshift = 0.2cm]
        (axis cs:8,5) -- (axis cs:13,5) node [black, midway, yshift = 0.8cm] 
        {\small \begin{tabular}{l}Excess supply \\ when price is $\$5$\end{tabular} };
  
      % Coordinate Points
      \addplot[color = black, mark = *, only marks, mark size = 2pt] 
        coordinates {(8, 5) (11, 4) (13, 4)};
      \node [above] at (11, 4.25) {\small $E$};
  
      % Labels
      \node [left] at (0, 3) {\small $\$3$};
      \node [left] at (0, 4) {\small $\$4$};
      \node [below] at (0, 0) {\small $0$};
      \node [below] at (8, 0) {\small $8$};
      \node [below] at (11, 0) {\small $11$};
      \node [below] at (13, 0) {\small $13$};
    \end{axis}
  \end{tikzpicture}
\end{frame}
% ------------------------------------------------------------------------------------------------

% ------------------------------------------------------------------------------------------------
\begin{frame}{Market Equilibrium}
  \textbf{Equilibrium} occurs at the price where quantity supplied equals quantity demanded. \alice{\emph{Why is this an equilibrium?}}

  \pause
  \begin{itemize}
    \item What would happen if the price is \$5 per bushel? \pause Excess supply will lead to a price war by suppliers.
      \begin{itemize}
        \item The price will fall until all units are sold.
      \end{itemize}

    \pause 
    \item What would happen if the price is \$3 per bushel? \pause Excess demand will lead to a bidding war by consumers.
      \begin{itemize}
        \item The price will rise until all consumers are satisified.
      \end{itemize}
  \end{itemize}


  \pause 
  Therefore, \$4 is an equilibrium because, absent any external forces, the price will not change.
\end{frame}
% ------------------------------------------------------------------------------------------------

% ------------------------------------------------------------------------------------------------
\begin{frame}{Calculating Market Equilibrium}
  \bgCranberry{Try It Yourself}

\bigskip
Suppose that the supply curve for Chaco's sandals in Boulder is given by $Q^S_{chaco} = -8,000 + 300P_{chaco}$, and the demand curve is $Q^D_{chaco} = 40,000  - 500P_{chaco}$. What is the equilibrium price of Chaco's?

\end{frame}
% ------------------------------------------------------------------------------------------------

% ------------------------------------------------------------------------------------------------
\begin{frame}{Aggregating Supply}
  Consider if you have two producers of a good. Each producer's supply curve tells us \emph{at a given price}, how many units will they sell.

  \bigskip
  How do we figure out the aggregate supply curve? 
  
  \pause 
  \begin{itemize}
    \item \emph{At a given price}, add up each producer's quantity supplied
  \end{itemize}
\end{frame}

\begin{frame}{Aggregating Supply}
  Suppose we have two producers, A and B. Suppose that
  $$
    Q_A^S =
    \begin{cases}
      4P - 12 & \text{if } P \geq 3 \\
      0       & \text{otherwise}.
    \end{cases} 
    \quad\quad
    Q_B^S =
    \begin{cases}
      3P - 15 & \text{if } P  \geq 5 \\
      0       & \text{otherwise}.
    \end{cases}
  $$

  \vspace{10mm}
  \only<1>{
    At a price of $P \geq 5$, both producers will supply, so supply is given by
    $$Q_{mkt}^S = Q_A^S + Q_B^S = (20 - 2P) + (21 - 3P) = 41 - 5P$$
  }

  \only<2>{
    At a price $3 \leq P < 5$, only producer A will produce ($Q_B^S = 0$), so supply is given by
    $$Q_{mkt}^S = Q_A^S + Q_B^S = (20 - 2P) + 0 = 20 - 2P$$
  }

  \only<3> {
    At a price $P < 3$, no one produces anything, so 
    $$Q_{mkt}^S = Q_A^S + Q_B^S = 0 + 0 = 0$$ 
  }
\end{frame}
% ------------------------------------------------------------------------------------------------

% ------------------------------------------------------------------------------------------------
\begin{frame}{Aggregating Supply}
  Putting this together, we have 
  $$
    Q_{mkt}^S =
    \begin{cases}
      7P - 27 & \text{if } P \geq 5     \\
      4P - 12 & \text{if } 3 < P \leq 5 \\
      0       & \text{otherwise}.
    \end{cases}
  $$
\end{frame}
% ------------------------------------------------------------------------------------------------

% ------------------------------------------------------------------------------------------------
\begin{frame}
  \bgCranberry{Try It Yourself}

  \bigskip
  Find the aggregate supply curve:

  $$
    Q_A^S =
    \begin{cases}
      5P - 25 & \text{if } P \geq 5 \\
      0       & \text{otherwise}.
    \end{cases}
    \quad\quad
    Q_B^S =
    \begin{cases}
      3P - 24 & \text{if } P  \geq 8 \\
      0       & \text{otherwise}.
    \end{cases}
  $$
\end{frame}
% ------------------------------------------------------------------------------------------------

% ------------------------------------------------------------------------------------------------
\begin{frame}{Shifts in Supply and Demand}
  Previously, we assumed that all factors other than price were fixed. But suppose that consumer incomes increase.
  \only<1>{\alice{\emph{What happens to the demand curve?}}}

  \bigskip
  \pause
  \begin{itemize}
    \item This causes the demand curve to shift to the right.

    \pause
    \item The price will rise, and the quantity sold will rise.

    \item Other causes of a demand shift are changes in preferences, the number of consumers, and expectations.
  \end{itemize}
\end{frame}
% ------------------------------------------------------------------------------------------------

% ------------------------------------------------------------------------------------------------
\begin{frame}[c]
  \begin{tikzpicture}
    \begin{axis}[
    width = 12cm,
    height = 9cm,
    xmin = 0, xmax = 25,
    ymin = 0, ymax = 9,
    axis lines = left,
    xtick = \empty, ytick = \empty,
    x label style={at={(axis description cs:1,0)},anchor=north east},
    y label style={at={(axis description cs:0,1)},anchor=south east},
    xlabel = {Quantity},
    ylabel = {Price},
    clip = false,
    ]
      % Demand Curve 1
      \addplot[color = alice!50!white, very thick] 
        coordinates {(0, 17/3) (11, 2) (14, 1) (17, 0)};
      \node [anchor = south west, text width = 3cm] at (17, 0) 
        {\small $D_1$};
        
      % Demand Curve 2
      \addplot[color = alice, very thick] 
        coordinates {(0, 23/3) (11, 4) (14, 3) (23, 0)};
      \node [anchor = south west, text width = 3cm] at (23, 0) 
        {\small $D_2$};
      
      % Supply Curve
      \addplot[color = ruby, very thick] 
      coordinates {(4, 1/2) (9, 3) (11, 4) (15, 6)};
      \node [anchor = south west, text width = 3cm] at (15, 5.5) 
        {\small $S$};



      % Coordinate Points
      \addplot[color = black, mark = *, only marks, mark size = 2pt] 
        coordinates {(8.5, 2.8) (11, 4)};
      \node [above] at (8.5, 3) {\small $A$};
      \node [above] at (11, 4.25) {\small $B$};

      % Curve shift arrow
      \draw[-stealth, very thick, alice] (axis cs:1.8, 5.5) -- (axis cs:5.5, 5.5);
  
      % Dotted lines
      \addplot[color = black, dotted, thick] 
        coordinates {(0, 4) (11, 4) (11, 0)};
      \addplot[color = black, dotted, thick] 
        coordinates {(0, 2.8) (8.5, 2.8) (8.5, 0)};

      % Axis Labels
      \node [anchor = south, rotate = 90] at (axis cs: 0, 3.375) 
      {\begin{tabular}{c}{\small Equilibrium price goes up} \\ $\rightarrow$\end{tabular}};

      \node [anchor = north] at (axis cs: 9.8, 0) 
      {\begin{tabular}{c}$\rightarrow$ \\ {\small Equilibrium quantity goes up}\end{tabular}};

    \end{axis}
  \end{tikzpicture}
\end{frame}
% ------------------------------------------------------------------------------------------------

% ------------------------------------------------------------------------------------------------
\begin{frame}{Shifts in Supply and Demand}
  Suppose now that the price of labor increases. \only<1>{\alice{\emph{What happens to the supply curve?}}}

  \bigskip
  \pause
  \begin{itemize}
    \item This causes the supply curve to shift to the left.

    \pause
    \item Price increases and quantity sold decreases.

    \item Other causes of a supply shift are changes in technology, input prices, number of suppliers, and expectations.
  \end{itemize}
\end{frame}
% ------------------------------------------------------------------------------------------------

% ------------------------------------------------------------------------------------------------
\begin{frame}[c]
  \begin{tikzpicture}
    \begin{axis}[
    width = 12cm,
    height = 9cm,
    xmin = 0, xmax = 25,
    ymin = 0, ymax = 9,
    axis lines = left,
    xtick = \empty, ytick = \empty,
    x label style={at={(axis description cs:1,0)},anchor=north east},
    y label style={at={(axis description cs:0,1)},anchor=south east},
    xlabel = {Quantity},
    ylabel = {Price},
    clip = false,
    ]
      % Demand Curve
      \addplot[color = alice, very thick] 
        coordinates {(0, 23/3) (11, 4) (14, 3) (23, 0)};
      \node [anchor = south west, text width = 3cm] at (23, 0) 
        {\small $D$};
        
      % Supply Curve 1
      \addplot[color = ruby!50!white, very thick] 
      coordinates {(4, 1/2) (9, 3) (11, 4) (15, 6)};
      \node [anchor = south west, text width = 3cm] at (15, 5.5) 
        {\small $S_1$};

      % Supply Curve 2
      \addplot[color = ruby, very thick] 
      coordinates {(0, 1/2) (5, 3) (7, 4) (11, 6) (15, 8)};
      \node [anchor = south west, text width = 3cm] at (15, 7.5) 
        {\small $S_2$};
     

      % Coordinate Points
      \addplot[color = black, mark = *, only marks, mark size = 2pt] 
        coordinates {(8.6, 4.8) (11, 4)};
      \node [above] at (11, 4.25) {\small $A$};
      \node [above] at (8.5, 5) {\small $B$};

      % Curve shift arrow
      \draw[-stealth, very thick, ruby] (axis cs:13.8, 5.5) -- (axis cs:10.2, 5.5);
  
      % Dotted lines
      \addplot[color = black, dotted, thick] 
        coordinates {(0, 4) (11, 4) (11, 0)};
      \addplot[color = black, dotted, thick] 
        coordinates {(0, 4.8) (8.5, 4.8) (8.5, 0)};

      % Axis Labels
      \node [anchor = south, rotate = 90] at (axis cs: 0, 4.4) 
      {\begin{tabular}{c}{\small Equilibrium price goes up} \\ $\rightarrow$\end{tabular}};

      \node [anchor = north] at (axis cs: 9.8, 0) 
      {\begin{tabular}{c}$\leftarrow$ \\ {\small Equilibrium quantity goes down}\end{tabular}};

    \end{axis}
  \end{tikzpicture}
\end{frame}
% ------------------------------------------------------------------------------------------------

% ------------------------------------------------------------------------------------------------
\begin{frame}{Shifts in Supply and Demand}
  What if \textit{both} curves shift simultaneously? Suppose, for example that demand increases but supply decreases.

  \begin{itemize}
    \item Both of these shifts result in a higher price.

    \item But they pull the equilibrium quantity in opposite directions

    \item We need to know the magnitude to know which direction quantity moves.
  \end{itemize}
\end{frame}
% ------------------------------------------------------------------------------------------------

% ------------------------------------------------------------------------------------------------
\begin{frame}
  \bgCranberry{Try It Yourself} 

  \bigskip
  Sketch a decrease in supply and an increase in demand where quantity goes up.
  Sketch a decrease in supply and an increase in demand where quantity goes down.
\end{frame}

\begin{frame}{Shifts in Supply and Demand}
  In general, when both curves shift, the change in either price or quantity will be obvious...

  \begin{itemize}
    \item ...but not both.

    \item One of these will always be ambiguous. Need to know magnitude of shifts.
  \end{itemize}

  \pause\bigskip
  Us professors love to ask questions on this. \textbf{When in doubt, draw it out!}
\end{frame}
% ------------------------------------------------------------------------------------------------

% ------------------------------------------------------------------------------------------------
\begin{frame}{Price Elasticity of Demand}
  Let's say your boss asks you what will happen to sales if they increase the price of their product. Don't you dare say ``sales will go down". 
  \emph{duh!!}

  \pause\bigskip
  We want to be able to predict \alice{how much} sales go down when we increase prices. \emph{Way more useful!}

  \pause\bigskip
  This is the \textbf{price elasticity of demand}
\end{frame}
% ------------------------------------------------------------------------------------------------

% ------------------------------------------------------------------------------------------------
\begin{frame}{Price Elasticity of Demand}
  \bgNavy{Price elasticity of demand} 
  
  Measures the sensitivity of the quantity demanded to changes in the price.

  $$
    \epsilon_{Q,P} = \frac{\text{percentage change in quantity}}{\text{percentage change in price}}
  $$

  \pause\bigskip
  Remembering our percent change formula
  $$
    \epsilon_{Q,P} = \frac{\frac{\Delta Q}{Q}}{\frac{\Delta P}{P} } = \frac{\Delta Q}{\Delta P} \frac{P}{Q}
  $$
\end{frame}
% ------------------------------------------------------------------------------------------------

% ------------------------------------------------------------------------------------------------
\begin{frame}{Interpreting Elasticities}
  \only<1>{
  $$
    \epsilon_{Q,P} = \frac{\frac{\Delta Q}{Q}}{\frac{\Delta P}{P} }
  $$
  }
  \only<2>{
  $$
    \epsilon = \frac{ \frac{\Delta \text{Top Thing}}{\text{Top Thing}} }{ \frac{\Delta \text{Bottom Thing}}{\text{Bottom Thing}} }
  $$
  }
  \bigskip

  \only<1>{
    A $1$\% increase in price yields a $\epsilon_{Q,P}$\% change in quantity. 
  }
  \only<2>{
    More generally, the way we \emph{always} interpret elasticities is:
    
    \bigskip
    A $1$\% increase in \alice{Bottom Thing} yields a $\epsilon$\% change in \alice{Top Thing}
  }
\end{frame}
% ------------------------------------------------------------------------------------------------

% ------------------------------------------------------------------------------------------------
\begin{frame}{Classifying the Price Elasticity of Demand}
\begin{center}
    \resizebox{\textwidth}{!}{
    \begin{tabular}{l l l}
      \toprule
      Value of $\epsilon_{Q,P}$ & Classification & Meaning \\
      \midrule
	  $\epsilon_{Q,P} = 0$ & \textcolor{navy}{Perfectly inelastic demand} & Quantity demanded is\\
	  & & completely insensitive to price \\
	  \addlinespace[0.5em]	  
	  $-1 < \epsilon_{Q,P} < 0$ & \textcolor{navy}{Inelastic demand} & Quantity demanded is\\
	  & & relatively insensitive to price\\	  
	  \addlinespace[0.5em]	  	  
	  $\epsilon_{Q,P} = -1$ & \textcolor{navy}{Unitary elastic demand} & Percentage increase in quantity\\
	  & & demanded is equal to percentage\\
	  & & decrease in price\\
	  \addlinespace[0.5em]	  
	  $-\infty < \epsilon_{Q,P} < -1$ & \textcolor{navy}{Elastic demand} & Quantity demanded is relatively\\
	  & & sensitive to price\\
	  \addlinespace[0.5em]	  
	  $\epsilon_{Q,P} = -\infty$ & \textcolor{navy}{Perfectly elastic demand} & Any increase in price results in\\
	  & & quantity demanded decreasing\\
	  & & to zero, and any decrease\\
	  & & in price results in quantity\\
	  & & demanded increasing to infinity\\
      \bottomrule
    \end{tabular}
    }
  \end{center}
\end{frame}
% ------------------------------------------------------------------------------------------------

% ------------------------------------------------------------------------------------------------
\begin{frame}{Calculating Elasticity}
  There's all kinds of elasticities we care about in economics. 

  \begin{itemize}
    \item The government might care what the price elasticity of demand is for cigarettes if they want to impose a tax.

    \item Or they might want to know what the cross-price elasticity of demand is for electric vehicles with respect to the price of gas.

    \item And if the government does impose some tax... firms might want to know how much of that tax they can pass on to consumers via higher prices.
  \end{itemize}
\end{frame}
% ------------------------------------------------------------------------------------------------

% ------------------------------------------------------------------------------------------------
\begin{frame}[c]
\begin{tikzpicture}
  \begin{axis}[
  width = 12cm,
  height = 9cm,
  xmin = 0, xmax = 120,
  ymin = 0, ymax = 7.5,
  axis lines = left,
  xtick = \empty, ytick = \empty,
  x label style={at={(axis description cs:0.5,-0.07)},anchor=north},
  y label style={at={(axis description cs:-0.07,.5)},anchor=south},
  xlabel = {\small Quantity},
  ylabel = {\small Price},
  clip = false,
  ]
    % Relatively Inelastics Demand Curve
    \addplot[color = alice, very thick] 
      coordinates {(0, 36/5) (15, 6) (40, 4) (90, 0)};
    \node [anchor = south west, text width = 3cm] at (42, -0.1) 
      {\small \begin{tabular}{l}Relatively \\ Inelastic \\ demand curve\end{tabular}};
      
    % Relatively Elatics Demand Curve
    \addplot[color = alice!50!white, very thick] 
    coordinates {(36,8) (38, 6) (40, 4) (44, 0)};
    \node [anchor = south west, text width = 3cm] at (88, -0.1) 
      {\small \begin{tabular}{l}Relatively \\ Elastic \\ demand curve\end{tabular}};
  
    % Dotted lines
    \addplot[color = black, dotted, thick] 
      coordinates {(0, 6) (38, 6) (38, 0)};
    \addplot[color = black, dotted, thick] 
      coordinates {(0, 4) (40, 4) (40, 0)};
    \addplot[color = black, dotted, thick] 
      coordinates {(15, 0) (15, 6)};

    % Coordinate Points
    \addplot[color = black, mark = *, only marks, mark size = 2pt] 
      coordinates {(15, 6) (38, 6) (40, 4)};

    % Labels
    \node [left] at (0, 6) {\small $\$6$};
    \node [left] at (0, 4) {\small $\$4$};
    \node [below] at (0, 0) {\small $0$};
    \node [below] at (15, 0) {\small $15$};
    \node [below] at (36, 0) {\small $38$};
    \node [below] at (42, 0) {\small $40$};
  \end{axis}
\end{tikzpicture}
\end{frame}
% ------------------------------------------------------------------------------------------------

% ------------------------------------------------------------------------------------------------
\begin{frame}
  \bgNavy{Income elasticity of demand} 
  
  \bigskip
  \% change in quantity demanded for every 1\% change in income.
  $$
    \epsilon_{Q,I} = \frac{\frac{\Delta Q}{Q}}{\frac{\Delta I}{I}} = \frac{\Delta Q}{\Delta I}\frac{I}{Q}
  $$

  \bigskip
  \bgNavy{Price elasticity of supply} 
  
  \bigskip
  \% change in quantity supplied for every 1\% change in price of good.
  $$
    \epsilon_{Q^S,P} = \frac{\frac{\Delta Q^s}{Q^s}}{\frac{\Delta P}{P}} = \frac{\Delta Q^s}{\Delta P}\frac{P}{Q^s}
  $$
\end{frame}
% ------------------------------------------------------------------------------------------------

% ------------------------------------------------------------------------------------------------
\begin{frame}
  \bgNavy{Cross-price elasticity of demand} 
  
  \bigskip
  \% change in quantity demanded of good $i$ for every 1\% change in price of good $j$.
  $$
    \epsilon_{Q_i,P_j}= \frac{\frac{\Delta Q_i}{Q_i}}{\frac{\Delta P_j}{P_j}}=\frac{\Delta Q_i}{\Delta P_j}\frac{P_j}{Q_i}
  $$
  
  \pause\bigskip
  If $\epsilon_{Q_i,P_j}>0$, then as $P_j$ increases, $Q_i$ increases.
  \begin{itemize}
    \item Then goods $i$ and $j$ are \textbf{substitutes}.
  \end{itemize}
  
  \pause\bigskip
  If $\epsilon_{Q_i,P_j}<0$, then as $P_j$ increases, $Q_i$ decreases.
  \begin{itemize}
    \item Then goods $i$ and $j$ are \textbf{compliments}.
  \end{itemize}
\end{frame}
% ------------------------------------------------------------------------------------------------

% ------------------------------------------------------------------------------------------------
\begin{frame}{Calculating Elasticity}
  \bgCranberry{Try It Yourself}

\bigskip
Suppose that when the price of car tires is \$100 per tire, quantity demanded in Detroit is 40,000. Now suppose that the price has fallen to \$90, and the quantity demanded is 50,000. What is the price elasticy of demand?


\end{frame}
% ------------------------------------------------------------------------------------------------

% ------------------------------------------------------------------------------------------------
\begin{frame}{Elasticity in the Long Run vs. the Short Run}
  Consumers can't always adjust their demand instantly in response to a price change.
  \begin{itemize}
    \item If the price of gasoline doubles, you still have to drive to work.

    \item But after a while, maybe you will buy a more fuel-efficient car.
  \end{itemize}

  \pause\bigskip
  Need to distinguish between \textbf{long-run demand curve} and \textbf{short-run demand curve}.
  \begin{itemize}
    \item Long run - Consumers have time to fully adjust purchasing decisions.

    \item Short run - Consumers do not.
  \end{itemize}
\end{frame}
% ------------------------------------------------------------------------------------------------

% ------------------------------------------------------------------------------------------------
\begin{frame}[c]
\begin{tikzpicture}
  \begin{axis}[
  width = 12cm,
  height = 9cm,
  xmin = 0, xmax = 120,
  ymin = 0, ymax = 7.5,
  axis lines = left,
  xtick = \empty, ytick = \empty,
  x label style={at={(axis description cs:0.5,-0.07)},anchor=north},
  y label style={at={(axis description cs:-0.07,.5)},anchor=south},
  xlabel = {\small Quantity (trillions of cubic feet per year)},
  ylabel = {\small Price (dollars per thousand cubic feet)},
  clip = false,
  ]
    % Short-run Demand Curve
    \addplot[color = alice, very thick] 
      coordinates {(0, 36/5) (15, 6) (40, 4) (90, 0)};
    \node [anchor = south west, text width = 3cm] at (42, -0.1) 
      {\small \begin{tabular}{l}Short-run \\ demand curve\end{tabular}};
      
    % Long-run Demand Curve
    \addplot[color = alice!50!white, very thick] 
    coordinates {(36,8) (38, 6) (40, 4) (44, 0)};
    \node [anchor = south west, text width = 3cm] at (88, -0.1) 
      {\small \begin{tabular}{l}Long-run \\ demand curve\end{tabular}};
  
    % Dotted lines
    \addplot[color = black, dotted, thick] 
      coordinates {(0, 6) (38, 6) (38, 0)};
    \addplot[color = black, dotted, thick] 
      coordinates {(0, 4) (40, 4) (40, 0)};
    \addplot[color = black, dotted, thick] 
      coordinates {(15, 0) (15, 6)};

    % Coordinate Points
    \addplot[color = black, mark = *, only marks, mark size = 2pt] 
      coordinates {(15, 6) (38, 6) (40, 4)};

    % Labels
    \node [left] at (0, 6) {\small $\$6$};
    \node [left] at (0, 4) {\small $\$4$};
    \node [below] at (0, 0) {\small $0$};
    \node [below] at (15, 0) {\small $15$};
    \node [below] at (36, 0) {\small $38$};
    \node [below] at (42, 0) {\small $40$};
  \end{axis}
\end{tikzpicture}
\end{frame}
% ------------------------------------------------------------------------------------------------

% ------------------------------------------------------------------------------------------------
\begin{frame}{Elasticity in the Long Run vs. the Short Run}
  The same is true for producers
  
  \begin{itemize}
    \item May not be able to increase output quickly in response to a price increase. Perhaps they are capacity-constrained.

    \item But in the long run, they can build another factory, or hire more workers. And quantity supplied increases.
  \end{itemize}
\end{frame}
% ------------------------------------------------------------------------------------------------

% ------------------------------------------------------------------------------------------------
\begin{frame}[c]
\begin{tikzpicture}
  \begin{axis}[
  width = 12cm,
  height = 9cm,
  xmin = 0, xmax = 350,
  ymin = 0, ymax = 27,
  axis lines = left,
  xtick = \empty, ytick = \empty,
  x label style={at={(axis description cs:0.5,-0.07)},anchor=north},
  y label style={at={(axis description cs:-0.07,.5)},anchor=south},
  xlabel = {\small Quantity (million megabytes per year)},
  ylabel = {\small Price (dollars per megabyte)},
  clip = false,
  ]
    % Short-run Supply Curve
    \addplot[color = ruby!50!white, very thick] 
      coordinates {(84, 2) (100, 10) (120, 20) (124, 22)};
    \node [above, text width = 3cm] at (124, 22) 
      {\small \begin{tabular}{l}Short-run \\ supply curve\end{tabular}};
      
    % Long-run Supply Curve
    \addplot[color = ruby, very thick] 
      coordinates {(25, 5) (100, 10) (250, 20) (265, 21)};
    \node [above, text width = 3cm] at (265, 22) 
      {\small \begin{tabular}{l}Long-run \\ supply curve\end{tabular}};
  
    % Dotted lines
    \addplot[color = black, dotted, thick] 
      coordinates {(0, 20) (260, 20)};
    \addplot[color = black, dotted, thick] 
      coordinates {(0, 10) (260, 10)};
    \addplot[color = black, dotted, thick] 
      coordinates {(100, 0) (100, 10)};
    \addplot[color = black, dotted, thick] 
      coordinates {(120, 0) (120, 20)};
    \addplot[color = black, dotted, thick] 
      coordinates {(250, 0) (250, 20)};

    % Coordinate Points
    \addplot[color = black, mark = *, only marks, mark size = 2pt] 
      coordinates {(100, 10) (120, 20) (250, 20)};

    % Labels
    \node [left] at (0, 10) {\small $\$10$};
    \node [left] at (0, 20) {\small $\$20$};
    \node [below] at (0, 0) {\small $0$};
    \node [below] at (90, 0) {\small $100$};
    \node [below] at (130, 0) {\small $120$};
    \node [below] at (250, 0) {\small $250$};
  \end{axis}
\end{tikzpicture}
\end{frame}
% ------------------------------------------------------------------------------------------------

% ------------------------------------------------------------------------------------------------
\begin{frame}{Elasticity in the Long Run vs. the Short Run}
  
  In some cases however, the opposite may be true.
  
  \bigskip
  If price falls for durable goods such as a new refrigerator, consumers may decide it's a good time to upgrade their old one.
  \begin{itemize}
    \item But in the end, they don't buy \textit{more} refrigerators. They just buy them sooner.
    
    \item In this case, demand is more elastic in the short run
  \end{itemize}

  
  \bigskip
  The same may be true of producers (such as in markets for used or recycled goods).
\end{frame}
% ------------------------------------------------------------------------------------------------

\end{document}
