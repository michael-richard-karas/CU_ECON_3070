\documentclass[11pt]{article}
\title{Recitation Problems - Ch. 2}
\input{preamble_recitation_worksheets.tex}

\begin{document}
  
\section*{Chapter 2}

\begin{enumerate}

	\item Suppose the market demand curve for a product is given by 
	$$
		Q^D_{mkt} = 1000 - 10P
	$$
	
	and the market supply curve is given by
	$$
		Q^S_{mkt} = -50 + 25P
	$$
	\begin{enumerate}
    \item What are the equilibrium price and quantity
    
    \item What is the inverse form of the demand curve?
  \end{enumerate}
  
  \newpage 
  \item Suppose that the individual demand curves for two individuals are given by
  $$
    Q^D_1 = \begin{cases} 
      200 - 10P & \text{if } P \leq 20 \\ 
      0         & \text{otherwise} 
    \end{cases}
    \quad\text{ and }\quad
    Q^D_2 = \begin{cases} 
      100 - 10P & \text{if } P \leq 10 \\ 
      0         & \text{otherwise} 
    \end{cases}
  $$
  and that the supply curve for the firm is given by
  $$
    Q^S_{mkt} = \begin{cases} 
      20P - 100 & \text{if } P \geq 5 \\ 
      0         & \text{otherwise} 
    \end{cases}.
  $$
  Find the market equilibrium (keeping in mind that the market price may be such that only 1 consumer will be willing to stay in the market).

  \newpage 
  \item Suppose that when Snarfburger originally charged a price of $\$5$ for their burger, they sold $1,000$ burgers per week. Thinking that they could potentially make more money by charging a higher price, they raised their price by $50$ cents. After raising their price, they sold $800$
  units per week. 
  
  \begin{enumerate}
    \item Find the price elasticity of demand for Snarfburgers. 
    
    \item Write a sentence interpreting the price elasticity of demand you calculated.
  \end{enumerate}
  
  \newpage
\begin{table}[htbp]
    \centering
    \caption{Estimates of the Price Elasticity of Demand for Selected Food Products}
    \resizebox{\textwidth}{!}{
    \begin{tabular}{lc}
        \toprule
        Product & Estimated $\epsilon_{Q,P}$ \\
        \midrule
        Cigars & -0.756 \\
        Canned and cured seafood & -0.736 \\
        Fresh and frozen fish & -0.695 \\
        Cheese & -0.595 \\
        Ice cream & -0.349 \\
        Beer and malt beverages & -0.283 \\
        Bread and bakery products & -0.220 \\
        Wine and brandy & -0.198 \\
        Cookies and crackers & -0.188 \\
        Roasted coffee & -0.120 \\
        Cigarettes & -0.107 \\
        Chewing tobacco & -0.105 \\
        Pet food & -0.061 \\
        Breakfast cereal & -0.031 \\
        \bottomrule
        \multicolumn{2}{p{\textwidth}}{\footnotesize Pagoulatos \& Sorensen (1986)}\\
    \end{tabular}
    }
\end{table}
	\begin{enumerate}
    \item Which good is the most inelastic?
    
    \item Which food product is more inelastic, Cheese or Roasted coffee?
    
    \item Is Ice cream considered an elastic good?
    
    \item Write a sentence interpreting the elasticity for Cookies and crackers.
  \end{enumerate}
\end{enumerate}

\end{document}